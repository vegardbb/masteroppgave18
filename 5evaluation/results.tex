\section{Testresultater} \label{testing}

Delkapitlet presenterer loggførte resultater fra tre tester der HTTP-forespørsler ble logget av DBUpgradinator og HTTP-responser ble logget av forespørselsgeneratoren på en datamaskin stasjonert i Trondheim.

\subsection{Test 1: Oppretting av aggregater, uten migrasjon}

I den første testen ble aggregater generert av frontendsimulatoren, beskrevet i delkapittel \ref{k51}, for deretter å bli sendt til hver sin tilfeldige applikasjonstjener i hver sin forespørsel, hvis metode er ''POST''. Denne testen ble kjørt for å måle omtrent hvor mange skriveforespørsler klyngen kan håndtere samtidig uten å migrere aggregatene som postes. Skriptet som simulerer applikasjons frontende velger pseudotilfeldig hvilken tjener som mottar hver HTTP-forespørsel, hvis metode er ''POST''. ''Frontenden'' logger også alle responsobjektene den mottar fra applikasjonstjenerne. Responsloggen er oppsummert i tabell \ref{frontend1}.

% Tabell som viser statistikk for HTTP-forepsørslene sendt i test 1
\begin{table}[hbtp]
  \begin{center}
    \begin{tabular}{ | l | r |}
      \hline
      \underline{Dato} & 2.06.2018 \\ \hline
      \underline{Varighet} & 2 t; 44 min; 18 sek \\ \hline
      \underline{Antall forespørsler} & 3587467 \\ \hline
      \underline{Antall 200-svar} & 3158592 \\ \hline
      \underline{Antall 500-svar} & 428875 \\ \hline
      \underline{Suksessandel} & 88.05 \% \\ \hline
      \underline{Feilandel} & 11.95 \% \\ \hline
    \end{tabular}
  \end{center}
  \caption{Tabb}
  \label{frontend1}
\end{table}


Denne testen ble avsluttet etter to timer og 44 minutter, av hensyn til størrelsen til loggdata som ble generert av forespørselskriptet og applikasjonstjenerne. Frontend\-programmet mottok i snitt omtrent 320 OK - responser per sekund. Litt over 11 prosent av forespørslene feilet.

% Tabell som viser statistikk for applikasjonsklyngen i test 1
\begin{table}[!h]
  \begin{center}
    \begin{tabular}{ | l | h |}
      \hline
      \underline{vbb-master2018-avery} & 791741 \\ \hline
      \underline{vbb-master2018-black} & 787906 \\ \hline
      \underline{vbb-master2018-crabbe} & 788961 \\ \hline
      \underline{vbb-master2018-dolohov} & 789984 \\ \hline
      \textbf{Sum aggregater} & \textbf{3158592} \\ \hline
    \end{tabular}
  \end{center}
  \caption{Persisterte aggregater på tjenere i testmiljøet under test 1.}
  \label{backend1}
\end{table}


Tabell \ref{backend1} viser at antallet '200 AOK' svar oppgitt i \ref{frontend1} samsvarer med det opptalte antallet persisterte objekter, naturlig nok fordi alle forespørsler kaller på samme metode. I snitt ble 520 aggregater skrevet til disk per sekund av de fire databasenodene. Til sammenlikning har Linkedin klart å betjene rundt 16 og et halvt tusen skriveforespørsler i sekundet på en klynge bestående av én eneste flertrådet tjener \citep{kreps2009}.

Den viktigste faktoren som bidrar til den forskjellen i gjennomstrømming mellom testmiljø og Linkedins resultater er quroum-konfigurasjonen. Både replikeringsfaktoren (N), lesefaktoren (R) og skrivefaktoren (W) til klyngen beskrevet i ''Performance''\-kapitlet er lik 1. De korresponderende innstillinger for klyngen DBUpgradinator ble testet i, er henholdsvis 3, 2, og 2.

I ''Performance'' - kapitlet til Project Voldemorts nettsider skriver \cite{kreps2009} at platelageret utgjør den ytelsesmessige flaskehalsen i en webapplikasjon. For å begrense feilraten kan man derfor øke størrelsen på hurtiglageret i server.properties-filene. Fordelingen av antallet persisterte aggregater blant de enkelte databasenodene er for øvrig brukbart jevn.

\begin{figure}[hbtp]
  \begin{turn}{-90}
    % Tabell som viser statistikk for applikasjonsklyngen i test 1
\begin{tikzpicture}
  \begin{axis}[xlabel={tid/s},ylabel={Persisterte aggregater},height=.91\textwidth,width=.91\textheight]
    \addplot[densely dotted] table[x=t,y=f,col sep=comma]{plots/post.txt};
  \end{axis}
\end{tikzpicture}

  \end{turn}{-90}
  \caption{Graf som plotter grupperte tidspunkt for persistering av data.}
  \label{plott1}
\end{figure}

Plottet i figur \ref{plott1} er basert på DBUpgradinators persisteringslogg for aggregatene som ble opprettet  i løpet av test 1. Loggtidspunktene har blitt gruppert sammen på sammenfallende sekund, og overført til en liste av (\emph{x},\emph{y}) - tupler der x indikerer antallet sekunder siden testen start og \emph{y} representerer antall aggregater persistert på tidspunkt \emph{x}.

\subsection{Test 2: Lesing, oppdatering, og oppretting av aggregater, med migrasjon}

I den andre testen forsøkte man å migrere aggregater postet under den første testen, gjennom både PUT-spørringer og GET-spørringer. Samtidig ble en stor mengde POST-spørringer, der det genererte aggregatet i forespørselens ''body'' følger den nye datamodellen, tilsendt klyngen. Testen var ment å simulere en migrasjonsprosess mens systemet betjente mange, hyppige forespørsler fra nye instanser av applikasjonen. På samme vis som i test 1 ble responsene fra tjenerne i produksjonsmiljøet logget til en fil.

% Tabell som viser statistikk for HTTP-forepsørslene sendt i test 2
\begin{table}[hbtp]
  \begin{center}
    \begin{tabular}{ | l | r |}
      \hline
      \underline{Dato} & 3.06.2018 \\ \hline
      \underline{Varighet} & 3 min; 1 sek \\ \hline
      \underline{Antall forespørsler} & 111914 \\ \hline
      \underline{Antall 200-svar} & 20871 \\ \hline
      \underline{Antall 500-svar} & 91043 \\ \hline
      \underline{Suksessandel} & 18.65 \% \\ \hline
      \underline{Feilandel} & 81.35 \% \\ \hline
    \end{tabular}
  \end{center}
  \caption{Tabell}
  \label{frontend2}
\end{table}

% Tabell som viser statistikk for applikasjonsklyngen i test 2
\begin{table}[hbtp]
  \caption{Persisterte aggregater hos databasenodene i testmiljøet under test 2.}
  \begin{center}
    \begin{tabular}{ | l | r |}
      \hline
      vbb-master2018-avery & 6149 \\ \hline
      vbb-master2018-black & 6286 \\ \hline
      vbb-master2018-crabbe & 5822 \\ \hline
      vbb-master2018-dolohov & 6233 \\ \hline
      \textbf{Sum aggregater} & \textbf{24493} \\ \hline
      Antall migrerte & 9206 \\ \hline
      Andel migrerte & 37.59 \% \\ \hline
      Antall brukergenererte & 15287 \\ \hline
      Andel brukergenererte & 62.41 \% \\ \hline
    \end{tabular}
  \end{center}
  \label{backend2}
\end{table}


Denne testen ble brått avbrutt etter bare tre minutter. Årsaken til at testen ble avbrutt så tidlig var at tre av databasenodene ble permanent utilgjengelige som følge av minneoverflyt i arbeidssettet deres. I alle tre nodene ble minneoverflyten utløst i et buffer for asynkrone spørringer, hvilket var forbeholdt migrasjonsspørringer.

På dette tidspunktet brukte \textbf{Migrator}-klassen grensesnittet \textbf{CompletableFuture} til å opprette tråder som kjørte migrasjonslogikken asynkront i en separat gruppe tråder. Etter dette testavbruddet ble kildekoden til DBUpgradinator endret, slik at kallene til CompletableFuture og asynkrone tilbakekall erstattet med \textbf{Thread} - konstruktøren og et kall til \emph{run}.

Tabell \ref{frontend2} viser en feilandel på åtte av ti sendte forespørsler. Dette kommer også av de fatale unntakene som ble kastet hos databasenodene. Tabell \ref{backend2} sier at andelen migrerte aggregater DBUpgradinator registrerte er under 40 prosent. Årsaken til dette er at forespørslene som poster nye aggregat for det nye skjemaet ikke behøver å migreres. I likhet med forrige test er skriveoperasjonene nesten jevnt fordelt blant de fire tjenernodene. I snitt ble 89 aggregat persistert på de fire forskjellige databasenodene per sekund.

\begin{figure}[hbtp]
  \begin{turn}{-90}
    % Tabell som viser statistikk for applikasjonsklyngen i test 1
\begin{tikzpicture}
  \begin{axis}[xlabel={tid/s},ylabel={Persisterte aggregater},height=.91\textwidth,width=.91\textheight]
    \addplot[densely dotted] table[x=t,y=f,col sep=comma]{plots/get-put-post.txt};
  \end{axis}
\end{tikzpicture}

  \end{turn}{-90}
  \caption{Graf som plotter grupperte tidspunkt for persistering av data (lesinger ekskludert).}
  \label{plott2}
\end{figure}

\subsection{Test 3: Oppdatering av aggregater, med migrasjon}

I den siste testen har en serie oppdateringsforespørsler (PUT) blitt sendt til applikasjonstjenerne til WebShopSimulator. Forespørsels\-generatoren har lest inn et sett med IDer som den mottok fra tjenerne i test 1. For å genere et oppdatert aggregat lagre den et helt nytt objekt jamfør framgangsmåten beskrevet i delkapittel \ref{k51}.

% Tabell som viser statistikk for HTTP-forepsørslene sendt i test 3
\begin{table}[!h]
  \begin{center}
    \begin{tabular}{ | l | h |}
      \hline
      \underline{Dato} & 4.06.2018 \\ \hline
      \underline{Varighet} & 1 t; 27 min; 56 sek \\ \hline
      \underline{Antall forespørsler} & 1019657 \\ \hline
      \underline{Antall 200-svar} & 703988 \\ \hline
      \underline{Antall 500-svar} & 315669 \\ \hline
      \underline{Suksessandel} & 69.04 \% \\ \hline
      \underline{Feilandel} & 30.96 \% \\ \hline
    \end{tabular}
  \end{center}
  \caption{Tabell}
  \label{frontend3}
\end{table}

% Tabell som viser statistikk for applikasjonsklyngen i test 3
\begin{table}[!h]
  \begin{center}
    \begin{tabular}{ | l | h |}
      \hline
      \underline{vbb-master2018-avery} & 355203 \\ \hline
      \underline{vbb-master2018-black} & 363147 \\ \hline
      \underline{vbb-master2018-crabbe} & 368200 \\ \hline
      \underline{vbb-master2018-dolohov} & 342810 \\ \hline
      \textbf{Sum aggregater} & \textbf{1429360} \\ \hline
      \underline{Antall migrerte} & 724831 \\ \hline
      \underline{Andel migrerte} & 50.71 \% \\ \hline
      \underline{Antall brukergenererte} & 704517 \\ \hline
      \underline{Andel brukergenererte} & 49.29 \% \\ \hline
    \end{tabular}
  \end{center}
  \caption{Table 1.}
  \label{backend3}
\end{table}


Ut ifra tabell \ref{backend3} kan man utlede at i snitt ble 133 aggregat persistert på de fire forskjellige databasenodene per sekund. Som forventet forekommer rundt én migrasjonsspørring for hver spørring som oppdaterer et aggregat. Også i denne testen er suksessraten for forespørslene for lav for produksjonsmiljø\-standarder. Kun sju av ti forespørsler ble besvart med en 200\-respons. Igjen er utilgjengelige databasenoder opphavet til statistikken, men av et annet slag enn hva tilfelle var i test 2.

Enhver databaseklient krever to bekreftelser på skriveoperasjoner før den returnerer en bekreftelse (skrive-ACK) til applikasjonen, og replikerer hvert aggregat til tre forskjellige noder (R = 3). For hver spørring applikasjonen gjør, gjør spørringskoordinatoren tre skrivespørringer. Når hver HTTP-forespørsel utløser en migrasjonsforespørsel på toppen av dette, blir arbeidslasten dobbelt så stor per aggregat. Hovedårsaken til at nodene gikk ned under denne testen, midlertidig om ikke annet, var at de ikke kunne respondere til flere TCP-forbindelser fra databaseklientene.

\begin{figure}[hbtp]
  \begin{turn}{-90}
    % Tabell som viser statistikk for applikasjonsklyngen i test 1
\begin{tikzpicture}
  \begin{axis}[xlabel={tid/s},ylabel={Persisterte aggregater},height=.91\textwidth,width=.91\textheight]
    \addplot[densely dotted] table[x=t,y=f,col sep=comma]{plots/put.txt};
  \end{axis}
\end{tikzpicture}

%  \begin{axis}[xlabel={tid/s},ylabel={Persisterte aggregater},height=.91\textwidth,width=.91\textheight,anchor=origin,rotate around={-90:(current axis.origin)}]

  \end{turn}{-90}
  \caption{Graf som plotter grupperte tidspunkt for persistering av oppdaterte data.}
  \label{plott3}
\end{figure}

