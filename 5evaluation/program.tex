% Tjeneste i fotnote: 
\section{WebshopSimulator} \label{prog}

Webapplikasjonen som DBUpgradinator testes på, kalles for \emph{WebshopSimulator}. Dens arkitektur følger både det logiske perspektivet, prosess\-perspektivet og den fysiske dittoen skissert i figurer \ref{fig5}, \ref{fig6}, \ref{fig7}, \ref{fig8}, \ref{fig9}, og \ref{fig10}. Testapplikasjonen etterkommer også systembegrensningene opplistet i delkapittel 4.2.2. Dette delkapitlet skisserer strukturen til testapplikasjonens kildekode, dets avhengigheter, og hvordan tjenerprogrammet blir kompilert til en kjørbar .jar - fil.

\subsection{Simulasjon av brukerforespørsler}

Frontend-delen av webapplikasjonen er skrevet i Javascript, og kjøres som et separat klientprogram. Dette skriptet kjøres i kjøretidsmiljøet NodeJS. Klientprogrammet kalles for \emph{WebAppSimulator}. Dets hovedoppgave er å simulere en kontinuerlig serie med forespørsler som jevnfordeles blant applikasjonsinstansene. Programmets funksjonalitet kan i korte trekk inndeles i tre satser:

\begin{enumerate}
  \item Programmet sender klyngen av tjenere en serie av POST-forespørsler, slik at nye aggregater lagres på hver av de tomme nodene i databasen. Totalt 8 GB i serialiserte JSON-objekter sendes applikasjonsinstansene og Voldemort-klientene. Samtidig holder en egen loggeprosess styr på ID-verdiene som genereres av tjenerne som tilsendes denne simulerte klienten
  \item Etter at den første satsen er ferdig, terminerer programmet. Da må applikasjonsloggen, generert av frontendsimulatoren, transformeres til en liste av unike IDer. Til å oppnå dette formålet brukes tekstbehandlingsverktøyet Vim, en kommandolinjeapplikasjon som kan redigere alle linjer i loggen samtidig ved hjelp av regulære uttrykk og dets kraftige kommandolinjesyntaks. Dernest må klientsimulasjonsprogrammet redigeres slik at listen av aggregat-IDer blir lest inn fra korrekt filsti, og slik at funksjonen som kjører den tredje satsen blir kjørt neste gang programmet blir startet opp fra kommandolinjen
  \item Programmet sender en blanding av 30 prosent PUT og 70 prosent GET - forespørsler til databasenodene, i sum én HTTP-forespørsel per ID i listen som ble oppretttet i forrige steg.
\end{enumerate}

For å generere data til POST - forespørslene i steg 1 og PUT - forespørslene i steg 3, brukes en komma\-separert seed-fil som inneholder en tabell der hver rad har en strengverdi for fornavn, etternavn, telefonnummer, gateadresse, poststed, delstat, postkode, og land\footnote{Navn - og addresser i seed-filen stammer fra gratistjenesten Fake Name Generator, som tilbyr inntil 100 000 navn og addresser i en kommaseparert fil per bestilling. URL: \url{https://www.fakenamegenerator.com/order.php}}. Simulasjonsprogrammet leser inn seedfilen og grupperer hver av kolonnene i tabellen til et sett med lister, der hver er assosiert med ett av de åtte tidligere nevnte kolonnenavnene.

