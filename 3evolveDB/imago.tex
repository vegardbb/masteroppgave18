\section{Imago}

% Innledning
Her følger en presentasjon av en helhetlig systemarkitektur som unngår versjonsløpsbetingelser og gjennomfører atomiske, konsistente datakonverteringsprosedyrer med minimalt behov for manuell styring. Med dette designet foreligger en vesentlig økning både i maskinberegninger (CPU), og bruk av lagringsressurser (disk). \cite{dumitras2009nodowntime} påpeker at designet kan innføre flere oppgradingsscenarier uten å påtvinge planlagt nedetid for applikasjonen som kjører med dette systemdesignet.

\cite{dumitras2009nodowntime} undersøker årsakene til planlagt nedetid ved å studere oppgraderingshistorikken til Wikipedia. Endringer i skjemaet som krever at både applikasjonen og datalaget oppgraderes i ett enhetlig, atomisk steg, fordrer vanligvis at hele programvaresystemet må slås av.

\subsection{Løsning} \label{imagolist}
Imago er bygget på tre grunnleggende målsetninger \citep{dumitracs2009upgrades}:
\begin{itemize}
  \item \textbf{Isolasjon}. Avhengigheter innad i det gamle unierset/versjonen av systemet må være adskilt fra oppgraderingsprosessen
  \item \textbf{Udelbarhet}. Under et hvilket som helst gitt tidspunkt kan klienter som kontakter systemet under oppgradering bli betjent av enten den gamle eller den nye versjon, men \underline{aldri} begge to. Dessuten må oppgraderingen skje på atomisk vis
  \item \textbf{Naturtro testmiljø}. Testmiljø må rekonstruere realistiske omgivelser som produksjonsmiljø trolig kan utsettes for
\end{itemize}

\cite{dumitracs2009upgrades} sitt bidrag til å automatisering av applikasjonsoppgraderinger er et system, der oppdateringer kan utføres online, og samtidig inkludere komplekse endringer i dataformatet. I Imago betegner begrepet ''univers'' en distinkt logisk mengde av tjenernoder som eksisterer i form av prosesser som kjører i fysiske datamaskiner eller prosesser i virtualiserte datamaskiner. Image definerer to slike univers, ett for det eksisterende systemet, og ett der den nye versjonen av systemet startes opp og kjører parallellt med den gamle versjonen av systemet. Baktanken med denne arkitekturen er å forhindre versjonsløpsbetingelser (''mixed version race''). \cite{dumitras2010upgrade} viser at versjonsløpsbetingelse kan medføre uante konsekvenser i systemets oppførsel i form av nye feil som oppstår. Disse feilene oppstår på grunn av uventet kommunikasjon mellom en oppgradert versjon av ett lag i systemet (for eksempel applikasjonslogikken) og en gammel versjon av ett annet lag, som for eksempel datalaget. Systemets arkitekter tar sjelden høyde for slike feil, gitt deres eksepsjonelle natur.

\subsection{Diskusjon}
% Kilde: Why Do Upgrades Fail? v/ Tudor Dumitras
En vesentlig antakelse som ligger til grunn for Imago er at det distribuerte systemet har ett aksesspunkt, eller API, for innkommende forespørsler fra brukere, og én eneste metode applikasjonstjenere kommuniserer med databasetjenere/datamodellen på.

% Styrker og svakheter : 2-2
Imago kan garantere fraværet av versjonsløpsbetingelser til gjengjeld for en vesentlig, dog midliertidig økning i bruk av lagringsressurser og beregningsressurser. KVolve har ennå til gode å bli implementert og testet ut på et større distribuert system \citep{saur2016}, men følger det samme udelbarhetsprinsippet som Imago.

% Fra Gramoli: Revider første setning
En svakhet med udelbar oppgradering er at oppgraderingsprosessen kan aldri termineres hvis systemet samtidig mottar mange datamanipuleringsforespørsler mens oppgraderingen kjører. Dette fordi en oppdatering øyeblikkelig fører til at alle tuplene som utgjør datamodellens nåværende tilstand, må leses på nytt. Derfor avbryter og utsetter Imago skriveforespørslene \texttt{CREATE}, \texttt{UPDATE}, og \texttt{DELETE} under oppgraderingsprosessen, enten ved å blokkere dem direkte, eller ved kun å tillate lesespørringer. For å sikre den atomiske oppgraderingen av systemet, er det nødvendig at utenforstående operasjoner ikke kan forstyrre datamigrasjoner forbundet til endring av applikasjonens oppførsel.
