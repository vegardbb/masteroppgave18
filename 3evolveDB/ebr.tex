\section{EBR}

Online oppgradering av distribuerte programvaresystemer blir stadig mer nødvendig for stadig flere virksomheter. Å planlegge nedetid for hele systemer,selv om det er for en god sak, nemlig viktige vedlikeholdsoperasjoner, er logisk ekvivalent med å avbryte flyten i forretningsprosessene og følgelig tape inntekter og kundetilfredshet -- med vilje vil kanskje utenforstående si. Behovet for online oppgraderinger er definitivt prekært, og en suksessfull implementasjon av automatisert oppgradering vil utvilsomt gi sterk datadrevne tertiære bedrifter et viktig konkurransefortrinn \citep{choi2009}.

Utgavebasert redefinisjon (EBR) er en oppgraderingsmetode ment for å oppgradere databaseprogrammet som holder rede på datalaget til en applikasjons arkitektur, uten å forårsake nedetid for applikasjonens sluttbrukere. Denne metoden må applikasjonens utviklere ha god kjennskap til når de implementerer selve oppdateringsskriptet, ellers er det vanskelig å få oppgradert applikasjonen online uten å påføre systemsvikt \citep{choi2009}.

EBR har en vært fast funksjonalitet i Oracle - databasesystemet siden versjon 11gR2, og ser ut til å være ganske populær blant store aktører med produksjonsmiljø som aksesseres av mange brukere verden over 24 timer i døgnet. \cite{choi2009} kommenterer at både BetFair og IFS, to store tertiære selskaper, er ivrige brukere av patching med EBR. Disse to kundene av Oracle demonstrerer at det er et meget stort marked for arkitekturer som støtter oppdatering av programvaresystem uten nedetid. Et kjapt internettsøk avslører at denne oppgraderingsmetoden er fremdeles ønsket og relevant for datadrevne applikasjoner som benytter en relasjonell datamodell. \footnote{En video der en ivrig EBR-bruker forklarer metodens fordeler er å finne på \url{https://www.youtube.com/watch?v=nRiAQgNDgoA}}

\subsection{Løsning}
I versjon 11.2 av Oracle definerer \cite{choi2009} til i alt tre nye konsepter for å muliggjøre online oppgradering:

\begin{itemize}
  \item Utgave (''Edition''). Hver instans av for eksempel en funksjon, SQL-setning, pakke eller view (immuterbar dataobjekt) eksisterer som en egen entitet, en utgave av nevnte konstruksjon. Slik definerer konseptet om utgaver en måte å skille variasjoner av det samme kodeobjektet fra hverandre, slik at man kan operere på en spesifikk instans av gangen
  \item Utgaveperspektiv (''Editioning View''). Ved hjelp av projeksjoner gjemmer utgaveperspektivet nyopprettede attributter eldre utgaver av koden ikke er ment å se fordi attributtet ikke skal eksistere i deres verdensbilde
  \item Triggere på tvers av utgaver (''Cross Edition Triggers''). For å synkronisere data forskjellige utgaver ikke har til felles, brukes triggere. Dataendringer som inntreffer hos tupler tilhørende eldre utgaver, propageres til kolonnene nye utgaver leser ifra, og vice versa
\end{itemize}

Meningen med utgaver er å sette opp et isolert miljø slik at mengden av kodeobjekter (instanser fra kildekoden) som trygt kan endres i tandem gjøres Slik. Å inkludere dataverdier i seg selv i dette miljøet vil åpenbart medføre mange unødige, tunge dataoperasjoner i den grad datatupler dermed må oppdateres og kopieres fram og tilbake for hver opp - eller nedgradering. Derfor behandles kolonner og tabeller som statiske elementer, åpent synlig for alle versjoner. Slik blir deling av data enklere. Semantisk sett må det gå raskt å opprette nye utgaver, men samtidig må tilstanden til koden pre-patch også være tilgjengelig for nye utgaver \citep{choi2009}. Utgaver innehar derfor følgende semantikk:

\begin{enumerate}
  \item Det finnes to typer objekter: versjonerbare (editionable) og ikke-versjonerbare, hvorav sistnevnte har den statiske egenskap at de er konstant identiske og synlige for alle utgaver (herunder inngår dataobjekter som tabeller og kolonner)
  \item Når en ny utgave opprettes, vil den holde på en komplett historikk av alle versjonerbare objekter
  \item Endringer av versjonerbare objekter er kun synlige for versjonen der nevnte endring ble utført
\end{enumerate}

For å illustrere hvordan de tre ovennevnte konseptene fungerer sammen for å realisere en trygg oppgraderingsprosess som kan brukes til online patching, så vel som tilbakerulling av oppgraderinger, beskriver \cite{choi2009} et case med et eksempelskjema levert av Oracle Corp. I caset blir et attributt for telefonnummer splittet i to, en for landskode og en for telefonnnummer innad i landet med den korresponderende landskoden. I den datamodellen utvinnes to nye attributter og én ny funksjonell avhengighet fra landskode på telefonnummer. I denne patchen blir også en del SQL-prosedyrer oppdatert for å samsvare med denne attributtsplittingen. Omfanget av dette caset representerer en typisk patch for en datamodell som allerede kjører i et produksjonsmiljø.

Tilbakerulling av en oppgradering i tilfelle en svikt oppsto med den nye utgaven er såre enkelt:
\begin{enumerate}
  \item Post-patch - utgaven blir slettet
  \item Nye tabeller, samt nye kolonner i eksisterende tabeller, som ble opprettet i løpet av oppgraderingen blir gjemt bort takket være projeksjonene til utgaveperspektiv-komponenten
\end{enumerate}

Ubrukte tabeller og kolonner kan dermed "tas fram igjen" \citep{choi2009} hvis man forsøker å installere patchen på nytt og lykkes. Tilbakerulling av oppgraderinger har ingen effekt på tilgjengeligheten til pre-patch - utgaven.

\subsection{Diskusjon}
\cite{choi2009} beskriver metoden utgavebasert redefinering (eng. Edition-Based Redefinition) som er implementert i Oracles database. EBR er en patchingmetode for databaseapplikasjon-instanser slik at patching av databasen ikke påkrever planlagt nedetid av applikasjonssystemet i sin helhet. Vitnesbyrd fra to store firma beviser at EBR fyller et sårt behov for online programvareoppgradering. Dataobjekter, som for eksempel tabeller, er inkompatible med utgavemetoden. EBR fokuserer på prosedyrer og funksjoner databasesystemet er bygget med. Med tanke på at artikkelen ble utgitt i 2009, er det tenkelig at de fleste industrier, deriblant telekom, allerede har anvendt manuell, rullerende oppgradering av sine distribuerte systemer over lang tid, og møtt på overraskende og dyre forviklinger som ikke kunne ha blitt forutsett uansett hvor erfarne databaseadministratorene er.

Som \cite{oliveira2006understanding} påpeker i deres oppsummering av undersøkelsene sine, er det menneskelige feil som står for majoriteten av feil som oppstår i databaser som kjører i et produksjonsmiljø. Feil som databaseadministratoren forårsaker, er som regel ikke maskerbare med for eksempel backup - teknikker eller redundansstrategier. Slike feil kan medføre blant annet at lagrede data blir fullstendig utilgjengelige, sikkerhetssårbarheter oppstår og at systemets ytelse svekkes betraktelig.

Et relevant spørsmål man kan stille denne oppgraderingsmetodologien er: Hvordan håndterer den miksede versjonstilstander? Det interessante svaret er at skopet til EBR er det relasjonelle databasesystemet applikasjonen persisterer sin datatilstand med. Versjonskappløp og miks av datalagversjoner og tjenestelagversjoner/controllerlagsversjoner er kun relevant for rammeverk som anskuer hele programvarearkitekturen og som versjonerer datamodellen separat fra selve applikasjonen, noe som faller naturlig for applikasjoner hvis datamodell opprettholdes av en relasjonell struktur.

Så kan man alltids lure på hvorfor \cite{choi2009} ikke stiller opp med en eksperimentell evaluering av teknologien de framstiller, all den tid designvalgene begrunnes blant annet med synet på ytelse. Kanskje oppleves skryten fra markedet eller det faktum at EBR fortsatt er en funksjonalitet av Oracle-databasen, i det vi skriver versjon 12c, som ble sluppet den 1. mars 2017, som bevis nok for bibliotekets utviklere. Dessuten er ordentlig testing av distribuerte systemer med en ordentlig distribuert arbeidslast en ganske dyr affære. Det er trolig grenser for hvor presise, kostbare og omfattende tester Oracle tillater for ett eneste aspekt ved Oracle RDBMS.
