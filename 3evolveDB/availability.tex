\subsection{Tilgjengelighet}

Et systems tilgjengelighet er kvalitetsattributtet som beskriver hvordan systemet oppfører seg i et abnormalt operasjonsmiljø, for eksempel ved et feilscenario eller et kontrollert avbrudd. Tilgjengelighet bygger på et annet kvalitetsattributt, pålitelighet, ved å legge til reparasjonsaspektet - når et system utsettes for en feil, så vil systemet respondere på hendelsen slik at feilen ikke forårsaker nedetid \citep{BCK2013}. Kvaliteten beskriver altså mer enn bare hvor ofte systemet er tilgjengelig for å behandle tjenesteforespørsler fra dets brukere.

\cite{BCK2013} definerer et programvaresystem sin tilgjengelighet som dets evne til å maskere eller reparere feil, slik at dets sum av alle perioder med nedetid innenfor et definert tidsintervall i forhold til nevnte intervall, ikke overstiger en bestemt prosentandel. Med ''nedetid'' menes en konkret, lukket tidsperiode der systemet ikke er mottakelig for kommando fra dets påtenkte brukere.

Mer presist beskrevet er steady-state-tilgjengelihet, en indikator på oppetiden til en komponent av et system eller systemet sett under ett, bestemt av to estimerte forventningsverdier: Tiden det tar før en svikt oppstår (MTBF), og reparasjonstiden (MTTR). Steady-state-tilgjengelihet er et forholdstall uttrykt matematisk som $ \frac{MTBF}{MTTR+MTBF} $.

I kontekst av programvareutvikling bør denne formelen tolkes dithen at for å diskutere systemets tilgjengelighet, må sannsynlige scenarier der feil oppstår og hvordan håndtere dem identifiseres. Effektene av hver feil må estimeres. Ikke minst må man tenke hvor lang tid man må bruke på å reparere eller maskere den enkelte feil. I tillegg kan feil unngås og fjernes, eventuelt kan man velge ikke å gjøre noe som helst i det de oppstår \citep{BCK2013}.

For å øke et systems tilgjengelighet må man minimalisere effekten som forårsakes av feil. En feil (eng. ''fault'') er et fenomen som forårsaker svikt i systemet (eng- ''failure''), skriver \cite{BCK2013}. En svikt er et avvik i systemets oppførsel fra dets spesifikasjoner. Implisitt i denne forklaringen av begrepet, ligger at systemsvikt er synlig for en menneskelig observatør, til eksempel en autorisert bruker. Eksistensen til en systemsvikt er altså ikke definert før den observeres.

Til vanlig telles ikke planlagt nedetid, som oppstår i forbindelse med vedlikeholdsoperasjoner som programvareoppgradering eller generell minnerensing ved restart av tjenere, i beregningen av tilgjengelighetsprosenten som typisk baseres på den tidligere nevnte formel. Vedlikehold utføres gjerne på et planlagt tidspunkt der forespørselstrafikken er på et svært lavt nivå. Forskningslitteratur dette dokument henviser til, indikerer at planlagt vedlikeholdsarbeid på et kjørende, distribuert system utgjør en egen klasse av mange forskjellige feilkilder som kan medføre nedetid, dog tradisjonelle feiltoleransemekanismer fokuserer på å motvirke, unngå, eller tillate uventede feil \citep{dumitracs2009upgrades}.

% Til tross for at denne detaljen opplyses i tjenestenivåavtalen der programvaresystemet er totalt utilgjengelig grunnet vedlikeholdsarbeid, men diverse brukere forespør dets tjenester likevel.

% TODO: BASE og PACELC
