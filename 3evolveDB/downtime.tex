\subsection{Nedetid forårsaket av vedlikeholdsarbeid}

En undersøkelse fra 1998 av totalt 426 høytilgjengelige programvaresystemer avdekket 75 \% av totalt 6000 tilfeller av systemnedetid kom av planlagt vedlikeholdsarbeid på enten maskinvare eller programvare, og disse planlagte nedetidsperiodene varte som regel dobbelt så lenge som de periodene med nedetid som kom av uventet systemsvikt \citep{lowell2004, dumitras2009nodowntime}. Feiltoleransemekanismer lages utelukkende for å kunne forhindre sistnevnte type nedetid. Dette har en sammenheng med at planlagt nedetid tradisjonelt ikke inngår i beregningen av total systemtilgjengelighet per år, fordi systemvedlikehold kunne gjøres på tidspunkt der systemets tjenester ikke etterspørres (for eksempel om natten). Høytilgjengelige systemer påkreves gjerne å være tilgjengelig 24 timer i døgnet, så denne betingelsen er naturligvis ikke lengre gjeldende.

Oppgradering av store informasjonssystemer er også svært dyre, grunnet større endringer i dataskjemaet/dataformatet og/eller datamigrasjon. Oppgraderinger som innebærer komplekse endringer i datamodellen medfører at dataene må gjennomgå tunge og langtekkelige konverteringer som kan ta flerfoldige timer for å konformere til nevnte endringer. Disse oppgraderingene er vanskelig å installere mens systemet leverer brukerne sine tjenester uten å forstyrre tjenesteleveransen såpass mye at systemets ytelse faller ned på uakseptable nivåer. Dette betyr at tunge oppdateringer i datamodellen ofte må gjøres offline, når systemet er slått av. Som en konsekvens av dette unnlater systemadministratorer i det lengste å endre datamodellen når applikasjonen er i et kjørende produksjonsmiljø \citep{dumitras2009nodowntime}.

\cite{dumitras2009nodowntime} hevder at manuelt styrte oppgraderingsmetoder som fordrer mye kunnskap om applikasjonslagets interaksjoner med datamodellen ikke er veien å gå for å nærme seg idealet om tilnærmet døgnåpen tjenesteytelse. Med sitt helhetlige system for gjennomføring av oppgraderinger med komplekse datamodellendringer, Imago, ønsker \cite{dumitras2009nodowntime} også å unngå behovet for å holde styr på avhengighetene til den kjørende/gamle versjonen. Det er jo tross alt et NP-hardt problem.

Ved å gjennomgå oppgraderingshistorikken til Wikipedia (\url{http://www.wikipedia.org}) påviser \cite{dumitras2009nodowntime} de viktigste årsakene til at nedetid blir planlagt ved systemoppgradering. Wikipedia er den mest ettertraktede kilden til informasjon på Internett. Leksikonet driftes på wiki-plattformen MediaWiki, implementert i PHP, som aksesserer en distribuert infrastruktur bestående av flere hundre databasetjenere.

\cite{dumitras2009nodowntime} stiller følgende eksempel for en skjemaendring (i en relasjonell database), illustrert i figur 2 på artikkelens tredje side: Kolonne \texttt{a} i tabell \texttt{X} blir byttet ut med kolonne \texttt{b}, begge kolonner er også representert i tabell \texttt{Y} hvis attributtverdier brukes til å initialisere attributtverdiene for \texttt{b} i tabell \texttt{X} med joinbetingelsen \texttt{Y.a = X.a}. Når kommandoen \texttt{DROP COLUMN a} kalles under \texttt{ALTER TABLE X}, vil påfølgende spørringer fra uvitende tjenere som kjører den gamle versjonen av MediaWiki påkalle databasefeil. På motsatt vis går det ikke an for oppgraderte MediaWiki-tjenere å spørre databasetjenere som kjører den gamle datamodellen der kolonne \texttt{b} ikke er definert i tabell \texttt{X}. Dermed må systemoppgraderingen foregå i to atomiske steg, der samtidige spørringer trygt kan komme innimellom: Først defineres kolonnen \texttt{X.b} og initialiseres med data fra \texttt{Y.b} ved hjelp av tidligere nevnte joinbetingelse. I samme steg oppgraderes applikasjonstjenerne til MediaWiki (dvs mengden av PHP-skript). Før dette steget er ferdig gjennomført får ikke klienter lov til å sende forespørsler overhodet. Steg 2: Slett kolonnen \texttt{X.a}.

Ved online oppgradering er det meningen at klienter kan aksessere systemet samtidig som det oppdateres. Den automatiserte prosedyren må derfor ta høyde for innkommende \texttt{UPDATE}, \texttt{INSERT} og \texttt{DELETE} - spørringer og samkjøre disse forespurte endringene med kolonnetillegget i tabell \texttt{X}. I praksis betyr dette at resultatrelasjonen fra joinbetingelsen blir kalkulert på nytt for hver nye forespørsel som inntreffer. \cite{dumitras2009nodowntime} argumenterer at sjonglering med versjonmiks mellom to forskjellige lag av systemet kan fort medføre ekstra ytelsesoverhead iform av forlenget oppgraderingstid sett i forhold til hvis den ble utført offline.

\cite{dumitras2009nodowntime} sin studie av MediaWiki sin oppgraderingshistorikk kan oppsummeres såeldes: Inkompatible skjemaendringer forhindrer rullerende oppgraderinger og påbyr at oppgraderingen skjer i ett helhetlig steg. Dataavhengigheter er vanskelige å synkronisere med endringer som følger av innkommende forespørsler og kan påføre ekstra prosesseringstid for oppdateringsprosessen samt overbelastning av lagringsressurser under datakonvertering og/ellere migrasjon. Dataskjemaendringer som motstrider hverandre, lik som flettekonflikter i asynkrone distribuerte databaser, krever manuell intervensjon.

% Revisjon: Er dette sant? I distribuerte miljøer, kan man lite på databasereplikering når man oppgraderer rullerende?
Når programvaren, eventuelt operativsystemet som kjører på hver enkelt datamaskin i et produksjonsmiljø spredt utover opptil flere datasentre på vidt forskjellige geografiske lokasjoner skal oppgraderes fra én versjon til en nyere, så har systemets administratorer flere valg. Man kan slå av alle noder i hele systemet, eventuelt hele datasentre om gangen, for deretter å installere oppdateringen på én og én node. I systemer med et moderat antall datamaskiner som tilsammen leverer en applikasjonstjeneste over et nettverk av klienter, og som alle er i ett og samme tjenerrom samlet i én klynge over et lokalt nettverk, vil denne såkalte ''stopp-verden'' - strategien, som \cite{saur2016} nevner, være en gangbar strategi som innebærer lave, tolererbare kostnader i form av tapt tjenestetid for brukerne. Til gjengjeld er man garantert at alle de oppdaterte 

% \cite{dumitras2009nodowntime} refererer til flere andre undersøkelser av systemer der feil har oppstått under systemoppgradering eller øvrig vedlikeholdsarbeid.

