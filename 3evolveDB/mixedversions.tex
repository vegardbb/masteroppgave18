\subsection{Versjonskappløp i distribuerte systemer}

% Mixed version race
\citep{dumitras2010upgrade} stiller følgende hypotetiske scenario om risiko forbundet med rullerende oppgradering av distribuerte kjøringsmiljøer: I en webbasert nettbankappliaksjon som komunniserer med et banksystem bestående av mange tjenere oppdages en potensielt farlig anomalie sikkerhetsmessig. I den gamle versjonen av klientprogramvaren finnes en tekstboks der brukeren skriver inn informasjon om pengeoverføring, både antall pengeenheter så vel som valutaen enhetene måles. Ergo aksepterer denne tekstboksen alle alfanumeriske tegn, hvilket betyr at nettbanken står i fare for å bli utsatt for SQL-innsprøyting gjennom web-grensesnittet.

For å motvirke denne sikkerhetsrisikoen beslutter ingeniørne som vedlikeholder programmet at den enkelte bruker skal velge pengevalutaen med en radio-boks istedet for å skrive den inn på tekstform. Dermed vil tesktboksen kun måtte akseptere tall. Imidlertid avdekkes en annen risiko under integrasjonstesting av denne feilreparasjonen: Denne endringen i klientdelen av arkitekturen medfører at tjenerapplikasjonen nå mottar to seperate parametre istedet for den ene strengen fra tekstboksen, og mens tjenerapplikasjonen oppgraderes, vil tjenere som kjører den utdaterte applikasjonen kun lese innholdet fra tekstboksen, som i den nye versjonen klientprogrammet bare er et tall, og følgelig anta på grunn av oppførselskonfigurasjon at beløpet som sendes er oppnevnt i amerikanske dollar. Denne inkonsistensen under oppgradering kan skape betydelig trøbbel for kunder som skal overføre penger i ikke-amerikanske valutaer.

Slike scenarier omtaler \citep{dumitras2010upgrade} som versjonskappløpsbetingelser ''mixed version race''. I hans artikkel ''To Upgrade Or Not To Upgrade'' demonstreres en sannsynlighetsmodell for risikovurdering av rullerende programvareoppgradering med spesielt hensyn på versjonskappløpsbetingelser, tilstanden systemet har der responsen på en forespørret operasjon er forskjellig mellom to noder som kjører hver sin versjon av nodeprogramvareprosessen.
