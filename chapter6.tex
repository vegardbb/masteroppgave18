%===================================== KAPITTEL 6 =================================

\chapter{Konklusjon og videre arbeid}

Dette kapitlet vil oppsummere masteroppgavens formål og hvordan det har blitt oppnådd. Oppfyllelse av målene satt i kapittel \ref{goals} blir vurdert.

I dette prosjektet ble en programvaremodul, kalt DBupgradinator, implementert i språket Java.

Testresultatene fra kapittel 5 demonstrerer at DBupgradinator kan brukes til å migrerere JSON-serialiserte dataobjekter til én versjon til den neste av en webapplikasjon, uten at noen noder blir slått av. Modulens største styrke er at den, gitt de antakelser presentert i kapittel 4, regulere versjonsmiks av skjemaer fordi de eksplisifiseres i applikasjonen. Modulens kildekode viser også at hvordan støtte for levende oppdatering av en aggregatoritentert datamodell implementeres i programkode avhenger av programvarens arkitektur.

Videre påfører datamigrasjonsmodulen, gjennom Migrator - klassen, én ekstra asynkron PUT - spørring som persisterer en ny utgave av et aggregat som har blitt ''opprettet'' av en aggregat-transformasjonsklasse og én ekstra GET - spørring i forkant av migrasjonen for å kontrollere at aggregatet ikke allerede er migrert til neste skjemaversjon. Sett fra applikasjonens perspektiv vil databaselaget i praksis yte færre brukerinitierte transaksjoner, det vil si PUT eller GET - operasjoner, per tid. Alternativt kunne implementasjonen droppe GET - spørringen til det nye potensielle aggregatet for å spare databasenodene for mange bomsøk som returnerer NULL. Implementasjonskapitlet beskriver at da vil PUT-spørringen som følger transformeringsklassens operasjon som oppretter en ny aggregatstreng potensielt overskrive en tidligere brukerinitiert transaksjon som inntreffer etter at skjemaet er oppgradert, eller i beste fall forårsake en divergering imellom replikaene transformasjonen skrives til.

Det er få artikler å finne i indeksen til Google Scholar på levende datamigrasjon av NoSQL-baserte applikasjoner, naturlig nok fordi datamigrasjon er en spesifikk problemstilling den enkelte tjenesteutvikler må ta stilling til og løse selv innenfor rammene satt av egne systemkrav, og derfor ikke spesielt interressant fra et akademisk synspunkt. 

Ett fremtidig mål for DBUpgradinator, som også ble nedfelt i \cite{saur2016} sine konklusjoner, er å definere et konsollprogram som opererer med et eget sett kommandoer som kan brukes til å redigere et konkretisert aggregatskjema, et domenespesifikt språk (DSL).  Formålet bak et slikt sett med denne spesialutformede syntaksen til konsollprogrammet som brukes til å endre strukturen til et aggregat er å strømlinjeforme prosessen med å definere oppgraderingsspesifikasjoner.

\cleardoublepage

