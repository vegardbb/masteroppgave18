\section{Krav til implementasjon av levende datamigrasjonsløsning}

Her følger de funksjonelle og ikke-funksjonelle krav som stilles til programvaremodulen som skal kunne migrere data fra et implisitt aggregatskjema til det neste uten stans i dataleveranser på brukernes tjenesteforespørsler.

\subsection{Antakelser}

I kraft av å demonstrere et praktisk konsept vil modulen gjøre visse antakelser om programvarearkitekturen til den jevne webapplikasjon DBUpgradinator kommer til å operere i. Enkelte av disse fordringene er gjort for å gjøre den objekt-orienterte implementasjonen av datamodellevolusjonsløsningen mindre parametrisert og derfor enklere og skrive og enklere å forstå.

For å realisere tillatelsen av versjonsmiks mellom forskjellige applikasjonstjenere i webapplikasjonen DBUpgradinator er installert i vil dets implementasjon ikke oppdatere aggregater direkte, men heller lage et nytt et med et annerledes streng-prefiks. Følgelig vil den samme ''tuppelen'' lagres dobbelt opp hvis miksen av skjemaversjoner er av størrelse 2.

Her listes øvrige antakelser, begrensninger og spesifikke designvalg for prototypeimplementasjonen av DBUpgradinator, databasen den kommer til å operere på, webapplikasjonen den testes i, og produksjonsmiljøet DBUpgradinator testes i.

\begin{itemize}
  \item I databasearkitektur modulen testes i serialiseres aggregater 'client' side, altså er lagringsnodene 'dumme' og backend-tjenerne 'smarte'
  \item Project Voldemort brukes som database i testapplikasjonen
  \item Serialisering (Voldemort): \underline{String}, altså ser den enkelte databasetjener kun strenger og kan ikke lese av skjemaet; videre antas det at databasen bruker LWW - strategien for å løse flettetvister automatisk
  \item Applikasjonsstakken til Voldemort-instansen denne oppgaven tester har en fast konfigurasjon
  \item Nøkler i skjemaet endrer ikke type eller form
  \item Dataobjekter har en skjema-versjonstag brukt til å sjekke om tuppelen må oppgraderes
  \item Fordi modulen i prinsippet må være skjemaopplyst må den implementeres som en del av applikasjonslogikken
  \item Oppdateringsspesifikasjoner kan kjedes sammen, denne kjeden har Migrator-klassen styr på, altså er det Migrator-klassen som har en tilstandvariabel som peker på en lenket liste av instanser av transformasjonsobjekter
  \item Akkurat som med KVolve blir ikke disse transformeringsobjektene fjernet når alle aggregatene lagret på det gamle skjemaet er migrert ferdig
  \item Applikasjonen, skrevet i Java, holder styr på sitt implisitte skjema gjennom eksplisitt versjonering og eksplisitt deklarering av aggregatets modell ved å definere en egen Aggregate-klasse
  \item Webapplikasjonens utviklere programmerer på et eget lokalt utviklingsmiljø
  \item I serversiden er tupler lagret med nøkler på form k:x, der suffikset x indikerer skjemaversjonen. Disse dataobjektene har ingen separate versjonsfelt idet de sendes Voldemort-serveren
  \item Den enkelte utvikler har ikke behov for å endre nøkkelens struktur eller datatype
  \item Alle nøkler Voldemort mottar er på strengform, selv om nøkler i praksis også kan være en liste av binærdata, derfor støtter ikke DBUpgradinator nedring av nøkkelens skjemastruktur
  \item Webapplikasjonen som modulen testes i har en RESTful - arkitektur
  \item For ett bestemt produksjonsmiljø er det til enhver tid kun én klientinstans som kommuniserer med de andre; denne klientinstansen sender til enhver tid transformasjonsobjekter i ordentlig rekkefølge, det vil si at klienten gjør seg ferdig med å sende ett transformasjonsobjekt for én skjemaversjon til alle applikasjonstjenere før denne sender over transformasjonsobjektet som gjelder for den etterfølgende versjonen
  \item Når transformasjonsobjekter sendes, så sendes de i en bestemt rekkefølge, altså er ikke rekkefølgen applikasjonstjenere mottar transformatorer i tilfeldig - denne antakelsen tillater oss å spesialisere og forenkle enkelte detaljer av kildekoden som kjører hos applikasjonsinstansene
  \item Modulen skal kjøre i produksjonsmiljøet til en webapplikasjon skrevet i Java
  \item Til enhver tid oppgraderes applikasjonen kun én versjon ad gangen, det vil si at på det meste er to forskjellige utgaver av versjonen som opererer i produksjonsmiljøet på ett og samme tidspunkt
\end{itemize}

\subsection{Funksjonelle krav}

\begin{enumerate}
  \item DBUpgradinator skal lage én ekstra databasespørring for hver nye transformasjon som påføres et aggregat
  \item Ved en GET-forespørsel til aggregatet med nøkkel \emph{x} fra en applikasjonsinstans som bruker en ny skjemaversjon \emph{k} skal DBUpgradinator også opprette en GET-forespørsel etter det samme aggregatet men med prefikset til den foregående skjemaversjonen i nøkkelen, \emph{j}
  \item Hver enkelt applikasjonsinstans har kontroll over versjonen av dataskjemaet sitt, som identifiseres ved en hashstreng som konkatineres med IDen til hver enkelt dataelement før PUT sendes til databasen
  \item DBUpgradinator vil opprette en ny tuppel i lageret hvis skjema-versjon-suffiks er annerledes og dataobjekt-ID-prefiks er likens prefikset til tuppelen som oppgraderes; således kan det distribuerte systemet kjøre en mikset tilstand mens datamigrasjonen foregår
  \item Ved kjøring av DBUpgradinator: Hvis programmet mottar en spørring etter et aggregat som tilhører den nyeste skjema-versjonen (f.eks en GET på en nøkkel hvis skjemaversjon-suffiks tilsvarer versjonsstrengen til den nyeste skjemaversjonen) så vil ikke en migrasjon med transformator-klassen bli utført
\end{enumerate}

\subsection{Kvalitetskrav}

\begin{enumerate}
  \item Tilgjengelighet: Det skal være mulig å forespørre data fra databasen mens datamigrasjon er iverksatt. Dette kravet evalueres ikke på kvantitativt, men på kvalitativt grunnlag, det vil si ved å kjøre DBUpgradinator i et simulert produksjonsmiljø, det vil si en applikasjon som ikke mottar forespørsler fra klienter som kommuniserer med HTTP.
  \item Ytelse: Ved rullerende, eller lat datamigrasjon, utsettes databasen for en vesentlig degradering i gjennomstrømskapasitet, målt i spørringer per sekund. Denne hemmelsen må være minimal i den implementerte migrasjonsløsningen.
  \item Modularitet: Løsningen som programmeres bør interferere minst mulig med eksisterende kildekode i Project Voldemort, primært gjennom å utvikle det som et separat prosjekt.
\end{enumerate}

