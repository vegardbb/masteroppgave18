\section{Krav til implementasjon av levende datamigrasjonsløsning}

Her følger de funksjonelle og ikke-funksjonelle krav som stilles til programvaremodulen som skal kunne migrere data fra et implisitt aggregatskjema til det neste, uten stans i dataleveranser på brukernes tjenesteforespørsler.

\subsection{Antakelser}

I kraft av å demonstrere et praktisk konsept vil modulen gjøre visse antakelser om programvarearkitekturen til den jevne webapplikasjon DBUpgradinator kommer til å operere i. Enkelte av disse fordringene er gjort for å gjøre den objekt-orienterte implementasjonen av datamodellevolusjonsløsningen mindre parametrisert og derfor enklere å skrive og enklere å forstå.

KVolve sitt design ikke tillater versjonsmiks aggregater imellom. Det gjør imidlertid DBUpgradinator. Når den migrerer et aggregat oppretter den en ny nøkkel i stedet for å skrive en migrasjon til samme nøkkel. Dette betyr at dataobjekter lagres dobbelt hvis to forskjellige versjoner av webapplikasjonen er i omløp i programvaresystemet.

Her listes øvrige antakelser, begrensninger og spesifikke designvalg for prototypeimplementasjonen av DBUpgradinator, databasen den kommer til å operere på, webapplikasjonen den testes i, og produksjonsmiljøet DBUpgradinator testes i.

\begin{itemize}
  \item Project Voldemort brukes som NoSQL-database i testapplikasjonen.
  \item I databasearkitekturen modulen testes i, serialiseres aggregater på 'klientsiden' av Voldemorts logiske arkitektur, som er å finne i ''Design'' - siden til \cite{kreps2009} - altså er lagringsnodene ''dumme'' og backend-tjenerne ''smarte''.
  \item Serialisering (Voldemort): \underline{String}, altså ser den enkelte databasetjener kun strenger og kan ikke lese av skjemaet; videre antas det at databasen bruker LWW - strategien for å løse flettetvister automatisk.
  \item Applikasjonsstakken til Voldemort-instansen som denne oppgaven tester, har en fast konfigurasjon.
  \item Nøkler i skjemaet endrer ikke type eller form.
  \item Dataobjekter har en skjema-versjonstag brukt til å sjekke om tuppelen må oppgraderes.
  \item Fordi modulen i prinsippet må være skjemaopplyst, må den implementeres som en del av applikasjonslogikken.
  \item Oppdateringsspesifikasjoner kan kjedes sammen, denne kjeden har Migrator-klassen styr på, altså er det Migrator-klassen som har en tilstandvariabel som peker på en lenket liste av instanser av transformasjonsobjekter.
  \item Akkurat som med KVolve blir ikke disse transformeringsobjektene fjernet når alle aggregatene som er lagret på det gamle skjemaet er migrert ferdig.
  \item Applikasjonen, skrevet i Java, holder styr på sitt implisitte skjema gjennom eksplisitt versjonering og eksplisitt deklarering av aggregatets modell ved å definere en egen \textbf{Aggregate}-klasse.
  \item Webapplikasjonens utviklere programmerer på et eget lokalt utviklingsmiljø.
  \item I databasen er aggregatet lagret med nøkler på formen \emph{k:x}, der suffikset \emph{x}. indikerer skjemaversjonen og prefikset \emph{k} indikerer applikasjonens egendefinerte ID for aggregatet. Disse dataobjektene har ingen separate versjonsfelt, idet de sendes Voldemort-serveren
  \item Den enkelte utvikler har ikke behov for å endre nøkkelens struktur eller datatype.
  \item Alle nøkler Voldemort mottar, er på strengform.
  \item DBUpgradinator støtter hverken endring av nøkkelens strengstruktur eller endring av nøkkelens datatype.
  \item Webapplikasjonen som modulen testes i har en RESTful - arkitektur.
  \item For ett bestemt produksjonsmiljø er det til enhver tid kun én klientinstans som kommuniserer med de andre: Denne klientinstansen sender til enhver tid transformasjonsobjekter i ordentlig rekkefølge, det vil si at klienten gjør seg ferdig med å sende ett transformasjonsobjekt for én skjemaversjon til alle applikasjonstjenere før denne sender over transformasjonsobjektet som gjelder for den etterfølgende versjonen.
  \item Transformasjonsobjekter sendes i en bestemt rekkefølge. Rekkefølgen applikasjonstjenere mottar transformatorer i er ikke tilfeldig. Denne antakelsen tillater oss å spesialisere og forenkle enkelte detaljer av kildekoden som kjører hos applikasjonstjenerne.
  \item Modulen skal kjøre i produksjonsmiljøet til en webapplikasjon skrevet i Java.
  \item Til enhver tid oppgraderes applikasjonen kun én versjon ad gangen, det vil si at på det meste er det to forskjellige utgaver av versjonen som opererer i produksjonsmiljøet på ett og samme tidspunkt.
\end{itemize}

\subsection{Funksjonelle krav} \label{funq}

\begin{enumerate}
  \item DBUpgradinator skal kunne migrere hvert aggregat i et aggregatorientert databasesystem som tilhører et utdatert implisitt skjema. For å gjøre det skal DBUpgradinator opprette en ny nøkkel hvis prefiks, det vil si strengen før kolonet, er uforandret, mens suffikset er endret til navnet på det nye skjemaet. På denne nøkkelen persisteres et nytt dataobjekt som er transformert fra et annet dataobjekt fra det gamle skjemaet.
  \item DBUpgradinator skal sende én ekstra skrivespørring til databasen for hver nye transformasjon som påføres et aggregat.
  \item Ved en GET-forespørsel til aggregatet med applikasjons-ID \emph{k} fra en applikasjonsinstans som bruker en gammel skjemaversjon \emph{x}, skal DBUpgradinator sende databasen en annen GET-forespørsel på en nøkkel opprettet under det nye skjemaet, \emph{y}. Dette gjøres for å kontrollere om aggregatet allerede har en korresponderende kopi i det nye skjemaet, før en eventuell migrasjon av aggregatet blir gjort.
  \item Hver enkelt applikasjonsinstans vet versjonen til det implisitte dataskjemaet sitt, som identifiseres ved en streng som DBUpgradinator kopler sammen med applikasjonens egendefinerte ID og adskiller dem med kolon, før en spørring sendes til databasen.
  \item For å migrere et aggregat med nøkkel \emph{k:x} vil DBUpgradinator opprette en ny nøkkel i databasen hvis skjemaversjon-suffiks er \emph{y} (den neste skjemaversjonen) og applikasjons-ID-prefiks er uforandret.
  \item Hvis DBUpgradinator mottar en spørring etter et aggregat som tilhører den nyeste skjema-versjonen (f.eks en GET på en nøkkel hvis skjemaversjon-suffiks tilsvarer versjonsstrengen til den nyeste skjemaversjonen), så vil ikke en migrasjon med transformator-klassen bli utført.
\end{enumerate}

\subsection{Kvalitetskrav}

\begin{description}
  \item[Tilgjengelighet] Det skal være mulig å forespørre data fra databasen mens datamigrasjon er iverksatt. Dette kravet evalueres på et kvalitativt grunnlag. For å teste dette kravet kjøres DBUpgradinator i et simulert produksjonsmiljø, det vil si en applikasjon som mottar forespørsler fra klienter som kommuniserer med HTTP.
  \item[Ytelse] Ved rullerende, eller lat datamigrasjon, utsettes databasen for en vesentlig degradering i gjennomstrømskapasitet, målt i spørringer per sekund. Denne hemmelsen må ikke være mer enn 50 \% i den implementerte migrasjonsløsningen.
  \item[Modulær løsning] Programmet bør kunne kjøres i de fleste ReST-orienterte webapplikasjoner skrevet i Java, uavhengig av hvilken aggregatorientert database applikasjonen lagrer data med.
\end{description}
