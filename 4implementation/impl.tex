\section{Implementasjon av DBUpgradinator}

DBUpgradinator er skrevet i det objektorienterte programmeringsspråket Java. Java er et statisk, nominelt, og sterkt typet språk. Med nominelt menes at type-ekvivalens eller type-kompabilitet baseres på eksplisitte deklarasjoner i kildekoden. I Java deklareres heltall representert ved fire bytes med nøkkelordet \emph{int}. Med sterk typing menes at en type, enten det er en primitiv datatype, streng eller hjemmelaget klasse, alltid blir deklarert for ethvert objekt i kildekoden. Et statisk typet språk er et språk der typen til en variabel kontrolleres idet kildekoden kompileres, ikke under kjøring.

Som vist i figur \ref{fig5}, består DBupgradinator av tre klasser (firkant): \textbf{AbstractAggregateTransformer}, \textbf{Migrator} og \textbf{StringQueryInterface}. Migrator - klassen innehar tre forskjellige migrasjonsmetoder, og to moduler (avrundet firkant): \emph{AppVersionResolver} og \emph{AggregateTransformerReceiver}. I implementasjonen er disse to elementene henholdsvis representert ved funksjonene \emph{getPersistedKey} og \emph{aggregateTransformerReceiver}. Koden i migrasjonsmetodene \emph{getAndMigrateAggregate}, \emph{putAndMigrateAggregate} og \emph{postAndMigrateAggregate} er skrevet i henhold til aktivitetsdiagrammene \ref{fig7}, \ref{fig8} og \ref{fig9}.

\subsection{AbstractAggregateTransformer}

\begin{lstlisting}[language=Java, label=ag, caption={Klassen \emph{AbstractAggregateTransformer}, hvis hovedansvar er å lage en ny streng ut fra et gitt strengargument i funksjonen \emph{transformAggregate}, som programvareutvikleren selv må implementere.}]
package vbb.dbupgradinator;

public abstract class AbstractAggregateTransformer {
    private String currentAppVersion;
    private String nextAppVersion;

    public AbstractAggregateTransformer(String currentAppVersion, String nextAppVersion) {
        this.currentAppVersion = currentAppVersion;
        this.nextAppVersion = nextAppVersion;
    }

    public String getNextSchemaVersion() {
        return nextAppVersion;
    }

    public String getAppVersion() {
        return currentAppVersion;
    }

    /**
     * Both parameters are strings either returned from or going into a DB query
     * @return String - the new aggregate based on the input
     * @param val The value, which is a generic object
    */
    public abstract String transformAggregate(String val);
}
\end{lstlisting}


\textbf{AbstractAggregateTransformer} er en abstrakt klasse, som har to konstante feltvariable. Verdien på disse feltvariablene defineres når dens konstruktør kalles. Den ene indikerer versjonen til skjemaet transformasjonsklassen mottar aggregater fra, og den andre indikerer versjonen til skjemaet som den transformerte verdien gjelder for.

Klassen har én abstrakt metode, som implementeres i en subklasse. Denne abstrakte metoden heter \emph{transformAggregate}, som påkalles når en forespørsel sendt fra en tjenerinstans som bruker den gamle skjemaversjonen, tilsendes databaseklienten. Oppgaven med å skrive denne subklassen overlates til applikasjonsutviklerne som benytter seg av DBUpgradinator. Etter å ha skrevet kildekoden til subklassen, må utvikleren distribuere .java - filen der subklassen er definert, til hver av applikasjonstjenerne manuelt. \texttt{.java} - filen må ligge underordnet samme katalog som den webapplikasjonens kildekode blir kompilert fra. Etterpå må klassen bli kompilert til en binær \texttt{.class}-fil.

% Sitat, lindholm1999: ''Two classes are the same class (and consequently the same type) if they are loaded by the same class loader and they have the same fully qualified name''
Den sterke, statiske typekontrollen til Java pålegger én lei begrensning for implementasjonen av migrasjonsmodulen, angående AAT. Det har seg slik at to klasser er identiske - hvis og bare hvis - de leses inn av samme klasseinnlaster (eng. ''class loader''), og har samme fulle navn \citep{lindholm1999}, i tillegg til å inneholde ekvivalent kildekode og like navn for alle variabler og metoder. Dersom de befinner seg i to forskjellige pakker, er de ikke identiske.

Det ble opprinnelig forsøkt å implementere et eget kommandolinjebasert program som lar utviklerne sende kompilerte \texttt{.class} - filer over TCP ved hjelp av klassene \textbf{ObjectOutputStream} og \textbf{ObjectInputStream} fra java.io - pakken. Ettersom den hjemmelagede aggregattransformatoren må være en spesialisering av AbstractAggregateTransformer, en klasse fra DBUpgradinator - modulen, så må artefakten DBUpgradinator.jar legges inn i klientprogrammets prosjektfiler som en avhengighet. En annen kopi av DBUpgradinator.jar er å finne blant webapplikasjonens avhengighetsfiler.

Per JVM-spesifikasjonen er klassen \textbf{AbstractAggregateTransformer} fra .jar - filen i klientprogrammet ikke den samme klassen som \textbf{AbstractAggregateTransformer} fra .jar - filen i tjenerprogrammet. Årsaken til det er at det er to forskjellige instanser av systemklasseinnlasteren som leser klassedefinisjonene inn i hver sin instans av JVM. Derfor ble ikke forsendelse av binære Java-klasser implementert.

Den enkleste løsning på dette problemet er å bruke getMethod() - metoden fra \emph{reflection}-biblioteket for å instansiere hver enkelt \emph{transformAggregate} - metode som en egen instans av \textbf{Method} - klassen. Ved migrasjon kalles \emph{invoke} - metoden for å kalle på \emph{transformAggregate}. Den returnerte verdien fra \texttt{invoke} - kallet kan da ''castes'' til \textbf{String}, jamfør linje 26 i kodeoppføring \ref{ag}. Denne løsningen skroter \textbf{Abstract}\-\textbf{Aggregate}\-\textbf{Transformer} - klassen fullstendig.

\subsection{Migrator} \label{migrator}

\begin{lstlisting}[language=Java, caption={Migrator-klassen i DBUpgradinator}]
package vbb.dbupgradinator;

import java.lang.reflect.Constructor;
import java.net.URI;
import java.nio.file.Files;
import java.nio.file.Path;
import java.nio.file.Paths;
import java.util.HashMap;
import java.util.concurrent.CompletableFuture;

import org.apache.logging.log4j.LogManager;
import org.apache.logging.log4j.Logger;

public class Migrator {
    private static final Logger logger = LogManager.getLogger("DBUpgradinator");
    private final String className;
    private final HashMap<String, AbstractAggregateTransformer> transformers = new HashMap<>(4, (float) 0.95);

    public Migrator() {
        this.className = "UserAggregateTransformer";
        // Sets up separate process which listens actively on the server socket
        new Thread( this::aggregateTransformerReceiver ).start();
    }

// Se kodelisting 4.2 til og med 4.7 for resten av kildekoden til Migrator
// ...

}
\end{lstlisting}


Migrator er en \texttt{public} klasse som eksponerer DBUpgradinator sin kjernefunksjonalitet til programvareutvikleren. Dens tilstand inkorpurerer aggregatets objekttype - det vil si domeneklassen til aggregatet, som sendes inn i transformasjonsfunksjonen til Abstract\-AggregateTransformer-klassen som verdi-argument. Transformasjonsfunksjonen blir overskrevet av den implementerte klassen. En instans av Migrator spør etter data på forespørsler tilsendt tjeneren med HTTP-protokollen, og persisterer samtidig migrerte aggregater returnert av transformasjonsfunksjonen til instansen av AbstractAggregateTransformer.

En instans av \textbf{Migrator} har en ordliste (\textbf{HashMap}) av instanser av \textbf{AbstractAggregateTransformer}, som utviklerens egenskrevne klasse AggregateTransformer (markert med rosa omriss i figur \ref{fig6}) arver fra. Det er Migrator-instansen som har ansvar for å utlede skjemaversjonskausalitet med denne lenkede listen og transformere innkommende aggregater, fra databasen så vel som applikasjonen, ved behov.

I likhet med Kvolve kan ikke en instans av Migrator slåes av, den er alltid på. Ei heller kan \textbf{AbstractAggregateTransformer} - objekter fjernes fra Migrators transformator\-ordliste, fordi migrasjonsprosessen ikke er styrt på tvungent vis, slik konvensjonen er med relasjonelle databaser. Følgelig kan man konkludere at man ikke kan være sikker på når en ''migrasjonsprosess'' med DBUpgradinator er omme i praksis. Dette kommer av at man da må overvåke alle nøkler i databasen på tvers av skjemaversjonene, hvilket ville ha vært lettere hvis det hadde vært mulig å utføre delvis nøkkeloppslag, på til eksempel skjemasuffikset, i nøkkelverdi\-lagre.

\begin{lstlisting}[language=Java, caption={Dynamisk metode i Migrator som kjøres i separat tråd.}]
// Addition of transformer class to the HashMap
private void addTransformer(AbstractAggregateTransformer t) { this.transformers.put(t.getAppVersion(), t); }

// Separate process that actively listens for new classes that extend AAT
private void aggregateTransformerReceiver() {
    // Define ClassLoader instance
    AggregateTransformerLoader loader = new AggregateTransformerLoader();
    while ( this.transformers.size() < 1 ) {
        try {
            Path path = Paths.get( new URI("./classes/" + this.className + ".class"));
            // Use Files.exists - condition
            if (Files.exists(path)) {
                byte[] classData = Files.readAllBytes(path);
                Class<?> c = loader.createClass(this.className, classData);
                Constructor cons = c.getConstructor(String.class, String.class);
                AbstractAggregateTransformer tran = (AbstractAggregateTransformer) cons.newInstance( "x", "y" );
                // What to do with the transformer object: Add it to the AAT list
                this.addTransformer(tran);
            }
        } catch (Exception e) {
            logger.error("An error occurred in AggregateTransformerReceiver ", e);
        }
    }
}
\end{lstlisting}


Funksjonen \emph{aggregateTransformerReceiver} vises i kodeoppføring \ref{atr}. Den påkalles hver gang en av migrasjonsfunksjonen blir kalt av applikasjonen (se oppføringer \ref{mget} og \ref{mput}), og én gang i konstruktøren til \textbf{Migrator}. Metoden gjør et I/O-kall til disk for å kontrollere om en ny \texttt{.class} - fil (med et spesifikt navn) er ankommet den oppgitte katalogen.

\begin{lstlisting}[language=Java, label=load, caption={Indre klasse brukt til å definere en transformasjonsklasse som arver konstruktør fra \textbf{AbstractAggregateTransformer}.}]
private class AggregateTransformerLoader extends ClassLoader {
    AggregateTransformerLoader() { super(); }
    final Class<?> createClass(String name, byte[] b) throws ClassFormatError {
        return super.defineClass(name, b, 0, b.length);
    }
}
\end{lstlisting}


I kodeoppføring \ref{load} kan man lese kildekoden til en indre klasse, \textbf{AggregateTransformerLoader}, som er en spesialisering av \textbf{ClassLoader}. En indre klasse er en ikke-statisk klasse, deklarert innen en annen klasse. Hver \textbf{Migrator} - instans har en egen instans av denne klassen. Den eneste oppgaven instansen har er å lese inn rekkefølger av oktetter (\textbf{byte[]} - typen i Java) og transformere dem til \textbf{Class} - objekter. Hver slik instans av \textbf{Class} brukes i sin tur til å instansiere \textbf{AbstractAggregateTransformer} - objekter.

\begin{lstlisting}[language=Java, caption={Funksjon som oppfyller rollen tl objektet ''AppVersionResolver'' fra figur \ref{fig5}.}]
private String getPersistedKey(String aggregateKey, String schema) {
    return aggregateKey + ":" + schema;
}
\end{lstlisting}


Oppgaven til funksjonen \emph{getPersistedKey} er å kople dataobjektets nøkkel i en innkommende HTTP-spørring (k) med applikasjonsversjonens nøkkel (x) på formen \texttt{k + ":" + x}. Det er en privat metode som brukes av de migrasjonsmetodene i kodeoppføringer \ref{mget} og \ref{mput}. Metoden er gjort privat for klasseinstansen med vilje for å abstrahere applikasjonen bort fra databaselogikken.

I Migrator - klassen er to metoder implementert for å migrere aggregater som kommer fra henholdsvis en \texttt{GET} - og en \texttt{PUT}\-forespørsel. I testapplikasjonen WebShopSimulator er det DBUpgradinator som står for all håndtering av spørringskall til databaseklienten.

\begin{lstlisting}[language=Java, caption={Metode for håndtering av GET-spørring i Migrator.}]
public String getAndMigrateAggregate(StringQueryInterface db, String aggregateKey, String schema) {
    String key = this.getPersistedKey(aggregateKey, schema);
    String aggregate = db.query(key); // Blocking DB op
    AbstractAggregateTransformer spec = this.transformers.get(schema);
    if (null != spec) {
        String nextSchema = spec.getNextSchemaVersion();
        String nextKey = this.getPersistedKey(aggregateKey, nextSchema);
        CompletableFuture.supplyAsync(() -> {
            if (!aggregate.equals("")) {
                // Migrate the aggregate having the key _key to another with the key _nextKey using spec
                String migratedAggregate = spec.transformAggregate(aggregate);
                Exception fail = db.persist(nextKey, migratedAggregate);
                if (fail == null) { return true; }
                logger.error("Error during migration from " + key + " to " + nextKey + ":\n"+ fail.toString());
            }
            return false;
        }).thenAccept((b) -> {
            if (b) {
                logger.info("Migrated aggregate with key " + key + " to " + nextKey);
            }
        });
    }
    return aggregate;
}
\end{lstlisting}


\texttt{GET}\-migrasjonsmetoden returnerer spørringsresultatet av brukerens forespørsel til applikasjonen. Selve migrasjonsprosessen blir utført ved å opprette én ny tråd der en lambda-funksjon blir kalt én gang.

\begin{lstlisting}[language=Java, caption={Metode for håndtering av PUT-spørring i Migrator.}]
public boolean migrateAndPutAggregate(StringQueryInterface db, String aggregateKey, String schema, String ag) {
    String key = this.getPersistedKey(aggregateKey, schema);
    AbstractAggregateTransformer spec = this.transformers.get(schema);
    if (null != spec) {
        String nextSchema = spec.getNextSchemaVersion();
        String nextKey = this.getPersistedKey(aggregateKey, nextSchema);
        CompletableFuture.supplyAsync(() -> db.persist(nextKey, spec.transformAggregate(ag))).thenAccept((fail) -> {
            if (fail == null) {
                logger.info("Migrated aggregate with key " + key + " to " + nextKey);
            } else {
                logger.error("Error during migration from " + key + " to " + nextKey + ":\n"+ fail.toString());
            }
        });
    }
    Exception ex = db.persist(key, ag);
    return logUpdateResult(key, ex);
}
\end{lstlisting}


Den andre migrasjonsfunskjonen er noe annerledes fra \emph{getAndMigrateAggregate}. Den returnerer en bekreftelse som indikerer hvorvidt en skrivespørring som ble utført i klassen som implementerer \textbf{StringQueryInterface} kastet et unntak eller ei. I fall \emph{put}-kallet gjorde det, returnerer migrasjonsfunksjonene \underline{false}, og \underline{true} i motsatt tilfelle.

\subsection{StringQueryInterface} \label{chaptersqi}

\begin{lstlisting}[language=Java, label=queryinter, caption={StringQueryInterface, grensesnittet Migrator-klassen bruker til å kjøre databasespørringer med.}]
package vbb.dbupgradinator;

public interface StringQueryInterface {
    public String query(String key);
    public Exception persist(String key, String aggregate);
}
\end{lstlisting}


\textbf{StringQueryInterface} er en liten mengde med funksjoner som utfører spørringer for Migrator-instansen. Kodeoppføring \ref{queryinter} viser hele dette grensesnittet, som eksponerer to metoder: \emph{query} og \emph{persist}. \emph{persist} - metoden forventes å returnere en instans av \textbf{Exception}. Grunnen til dette er at \emph{persist} skal kalle \emph{put}-metoden til en instans av \textbf{StoreClient}. I Voldemort sitt tilfelle kan \emph{put} kaste et \textbf{ObsoleteVersionException} - unntak, som må fanges opp. 

For at \textbf{Migrator} - instansen skal kunne registrere at spørringen feilet, må den motta et unntak som retur\-verdi. Den verdien kommer til å være lik \texttt{null} hver gang spørringen ble fullbyrdet. Utvikleren forventes å skrive en separat klasse som implementerer dette grensesnittet. Det er denne klassen som må rydde opp i eventuelle flettekonflikter, det vil si hver gang \emph{get}-kallet returnerer en liste av aggregater i stedet for bare ett.

Som en konsekvens av denne kildekoden, kan DBUpgradinators brukes i webapplikasjoner med aggregatorienterte datamodeller som bruker andre nøkkelverdi-lagre enn Project Voldemort til å persistere data.

