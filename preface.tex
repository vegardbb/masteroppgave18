\section*{\Huge Forord}
\addcontentsline{toc}{chapter}{Forord}
$\\[0.5cm]$

Min masteroppgave presenterer et modulært programvarebibliotek som automatiserer oppdatering av semistrukturerte datamodeller i distribuerte, aggregatorienterte databasesystemer. Rapporten utgjør min besvarelse som vurderes i emnet TDT4900 - Datateknologi, masteroppgave, og utgjør samtidig mitt siste innleveringsarbeid i studieprogrammet MTDT - Datateknologi ved Norges Teknisk - Naturvitenskapelige Universitet i Trondheim. Oppgaven er basert på vitenskaplige kilder funnet og diskutert i løpet av fordypningsprosjektet jeg gjennomførte høsten 2017.

Formålet med oppgaven er å utforske hvordan prosessen med å oppgradere moderne webapplikasjoner som allerede kjører i et fungerende, aktivt produksjonsmiljø uten å slå av tjenesten. En egen løsning for denne problemstillingen er blitt implementert og testet i et realistisk oppgraderingsscenario for en typisk datamodell i en ekommersiell setting.

En stor, personlig takk rettes til min veileder Svein Erik, for gode, motiverende svar på mine spørsmål og usikkerheter rundt dette prosjektet, samt frie tøyler til å forme masteroppgaven etter eget ønske.

Rapporten er skrevet i \LaTeX, og benytter en mal laget av Agus Ismail Hasan. \footnote{Malen er tilgjengelig fra DAIM sin FAQ, \url{https://daim.idi.ntnu.no/faq_innlevering.php}} Takket være hans arbeid med denne malen sparte jeg mye tid på å sette opp dokumentets tekniske struktur, og det er derfor forfatteren krediteres i dette forordet.

Jeg vil også takke min tante, forhenværende lærer og utdannet logoped Nella Lovise Bugge, for hjelp med korrekturlesing av denne prosjektrapporten.

% Benytt høyrejustering her
\begin{flushright}
Trondheim, \today

Vegard Bjerkli Bugge
% Slutt på høyrejustering
\end{flushright}

\cleardoublepage
