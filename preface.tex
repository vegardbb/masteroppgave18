\section*{\Huge Fordord}
\addcontentsline{toc}{chapter}{Fordord}
$\\[0.5cm]$

\noindent Min masteroppgave presenterer et modulært programvarebibliotek som automatiserer oppdatering av semistrukturerte datamodeller i distribuerte, aggregatorienterte databasesystemer. Rapporten utgjør min besvarelse som vurderes i emnet TDT4900 - Datateknologi, masteroppgave, og utgjør samtidig mitt siste innleveringsarbeid i studieprogrammet MTDT - Datateknologi ved Norges Teknisk - Naturvitenskapelige Universitet i Trondheim. Oppgaven er basert på vitenskaplige kilder funnet i løpet av fordypningsprosjektet jeg gjennomførte høsten 2017.

\noindent Formålet med oppgaven er å utforske hvordan prosessen med å oppgradere moderne webapplikasjoner som allerede kjører i et fungerende, aktivt produksjonsmiljø uten å slå av tjenesten. En egen løsning for denne problemstillingen er blitt implementert og testet i et realistisk oppgraderingssscenario for en typisk datamodell i en ekommersiell setting. Opp igjennom det siste tiåret har det vært vanlig å oppgradere programvare som kjører i et system av flere instanser, eller prosesser, på rullerende vis. I denne manuelt kontrollerte oppgraderingsmetoden blir én etter én instans av den gamle versjonen av programmet avsluttet og erstattet med en instans av den nye versjonen. Et vesentlig problem med denne metoden er at applikasjonens datamodell er som regel realisert i et database

\noindent Den enkelte leser behøver ikke ha noen dype forkunnskaper om datamaskinvare eller operativsystemer. Det antas imidlertid at leseren er kjent med fenomenet ''prosess'' i kontekst av operativsystemer, samt tradisjonelle databasesystemer, transaksjonsmønsteret og dets fire kvalitative egenskaper.

\noindent En stor, personlig takk rettes til min veileder Svein Erik, for gode, motiverende svar på mine spørsmål og usikkerheter rundt dette prosjektet, samt frie tøyler til å forme masteroppgaven etter eget ønske.

Rapporten er skrevet i \LaTeX, og benytter en mal laget av Agus Ismail Hasan. \footnote{Malen er tilgjengelig fra DAIM sin FAQ, \url{https://daim.idi.ntnu.no/faq_innlevering.php}} Takket være hans arbeid med denne malen sparte jeg mye tid på å sette opp dokumentets tekniske struktur, og det er derfor forfatteren krediteres i dette forordet.

\noindent Jeg vil også takke min tante, forhenværende lærer og utdannet logoped Nella Lovise Bugge, for hjelp med korrekturlesing av denne prosjektrapporten.

% Benytt høyrejustering her
\begin{flushright}
Trondheim, \today

Vegard Bjerkli Bugge
% Slutt på høyrejustering
\end{flushright}

\cleardoublepage
