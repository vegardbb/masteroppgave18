\section{Prosjektets mål}

Dette fordypningsprosjektet har hatt følgende overordnede mål:

\begin{enumerate}
  \item Utdype hva det vil si å gjøre programvareoppdateringer ''online'' og rullerende oppgraderinger % kap 1 og 2
  \item Sette opp en oversikt over tilgjengelig vitenskaplig literattur og øvrige nettressurser utgitt \\ av de som drifter storskale internettjenester om online oppgradering i storskala webapplikasjoner
  \item Drøfte eksisterende teknikker/arkitekturer/prosesser som implementerer \\ online oppgradering i en distribuert kontekst og vurdere hvordan hver av disse løser problemet
\end{enumerate}

Denne rapporten diskuterer også hvordan rullerende oppgradering av et distribuert nøkkel-verdi-databasesystem kan i så stor grad som mulig automatiseres. Hva kan evt. gjøres for å utbedre dette for én av databasene?


%--------------------------------------------------


%\subsubsection{Kladd for foreløpig problemstilling}

%''High-avalibility og key-value/NoSQL-databaser. Evaluer hvordan NoSQL-databasen støtter høytilgjengelig. Hvilken støtte har de for online upgrades og andre online maintenance-oppgaver? ''\\

%\textit{Oppgaveforslag}

%I dette prosjektet har det blitt foretatt et litteraturstudium på hvordan programvareoppgraderinger gjøres i databaseklynger (cluster). Det belyses hvilke mønstre og teknikker benyttes i industrien

%Konkrete forslag til hvordan en av disse teknikkene kan implementeres i et høytilgjengelig databasesystem

%\texttt{TBC, se siste side av ''To Upgrade or not'' og viko for info om problemstillingdefinisjon.}
%Prosjektets bidrag (10.09.17)

%Undersøker systemets støtte for rullerende oppgradering av nodeprosesser, under forskjellige betingelser/begrensninger (e.g. at oppdateringen medfører kaskade-endringer/propagerer nye endringer)

%Konkrete forslag til hvordan forbedre støtte for rolling upgrades.

%\\ Øvrige innfallsvinkler
%\textit{Vinkel 1}

%I dette prosjektet skal systemtilgjengeligheten til fire forskjellige distribuerte databasehandteringssystemer (DBMS) som er del av NoSQL - økosystemet undersøkes. Hvordan oppfører de distribuerte databasesystemene seg når en node (programvareprosess, ikke datamaskin) avsluttes og erstattes med en ny etter en viss periode (\texttt{dette scenariet er vanlig - bør her vise til kilder fra industrien}), og er det en optimal prosess gitt omstendighetene (som clusteret er blind for) ?

%Notat: Et spesielt tilfelle av kontrollerte avbrudd er oppdateringer av dataskjemaet der typekodingen, for eksempel konvertering fra ASCII - koding til UTF - koding. Her kan ikke persistert data brukes som før, men må istedet oversettes til det nye kodeformatet.\\

%\textit{Vinkel 2}

%I dette prosjektet skal et verktøy for handtering av \underline{kontrollerte tjeneravbrudd} (spesielt ord som bør defineres bedre) implementeres og testes. \textbf{Spørreord: Hvem? Jeg; Hva? Hvorfor? Hvilke? Når?}

%\newpage
