\section{Oppgavens problemstilling og mål}

Denne masteroppgaven opererer med en konkret definert problemstilling. Her presesnteres denne definisjonen, hvorpå et sett med konkrete, oppnåelige, og tidsbestemte målpunkter forbundet til definisjonen av problemstillingen også blir listet opp.

\textbf{Problemstilling:} Formålet med dette prosjektet er å implementere et høytilgjengelig programvaresystem bygget med en nøkkel-verdi-datamodell realisert med databasehandteringssystemet Project Voldemort, der oppgraderinger som involverer endringer 

Oppgaven har hatt følgende overordnede mål:

\begin{enumerate}
  \item Beskrive sammenhengen mellom kontinuerlig programvareleveranse og levende oppgradering av datamodeller % kap 1 - 3
  \item Modellere en modulær løsning der semistrukturerte datamodeller, også referert til som NoSQL-datamodeller, kan oppgraderes synkront med applikasjonstjenerne
\end{enumerate}


%---------------------
%Notater

% For this master thesis, a set of goals has been determined including a problem definition. The definition is first presented, and then some formalized goals related to the definition is given. The problem can be defined as following:

% Problem definition. The purpose of this thesis is to evaluate the possibility of incorporating the join operation in a NoSQL system. This includes how to formulate joins in the query language or API. We would like to know if the lack of join operation in NoSQL databases is due to architectural decisions, making join hard to implement, or if it is straightforward to include it. The work will focus on a single node since distributed join is considered hard to implement, and will not bring forward issues which are specific to NoSQL databases only.
