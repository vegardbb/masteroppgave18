\section{Oppgavens struktur}

Denne rapporten har følgende struktur. Kapittel 1 er introduksjonskapitlet, som illustrerer problemstilllingen som undersøkes og kort hvordan. Det beskrives hvordan litteratursøket ble gjennomført, samt hvordan analysen av de ulike oppgraderingsmetodene utføres.

Kapittel 2 omtaler relevant teori, herunder en kort innføring i databasearkitektur, distibuerte systemer, tilgjengelighetskvaliteten til et system, oppgradering av programvaren som utgjør noder i distribuerte systemer og nedetid i systemer som oppstår i forbindelse med programvareoppgradering av dem. Konseptet rullerende oppgradering defineres her. I teorikapitlet presenteres også relevant arbeid gjort på dette feltet.

I kapittel 3 sammenliknes løsningsforslagene artiklene kommer med for online oppgradering av distribuerte databasesystem. Denne evalueringen er ''fortrinnsvis'' kvalitativ (det vil si at det har vist seg komplisert å produsere et optimalt datagrunnlag for en kvantitativ test) der vurderingskriteriene bunner i hvor tilgjengelig løsningen er for det allmenne marked, hvor forståelig publikasjonene som presenterer er for rapportens forfatter og popularitet i industrien - det er jo tross alt et problem av industriell udertone som besvares her.

I kapittel 4 konkluderes evalueringen og i det beskrives forslag til videre kvantitativt feltarbeid som kan bygge på denne analysen.
