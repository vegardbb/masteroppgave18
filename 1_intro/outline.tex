\section{Oppgavens struktur}

Denne rapporten har følgende struktur. Kapittel 1 er introduksjonskapitlet, som illustrerer problemstilllingen som undersøkes og kort hvordan. Det beskrives hvordan litteratursøket ble gjennomført, samt hvordan analysen av de ulike oppgraderingsmetodene utføres.

Kapittel 2 omtaler relevant teori om NoSQL - datamodeller, en kort innføring i databasearkitektur, distibuerte systemer, tilgjengelighetskvaliteten til et distribuert programvaresystem, oppgradering av programvaren som utgjør noder i distribuerte systemer og nedetid i systemer som oppstår i forbindelse med programvareoppgradering av dem. Konseptet ''levende oppgradering av programvaresystemer'' defineres og forklares her. I teorikapitlet presenteres også NoSQL - DBMSet Project Voldemort, som vedlikeholdes av et dedikert utviklingslag hos LinkedIn.

Det tredje kapitlet beskriver et knippe oppgraderingsverktøy som på hvert sitt eget vis kommer i bukt med versjonmiks - problemet beskrevet i kapittel 1.1. I kapittel 3 sammenliknes løsningsforslagene for online oppgradering av distribuerte databasesystem, som ble lest om . Denne sammenliknignen er ''fortrinnsvis'' kvalitativ (det vil si at det har vist seg komplisert å produsere et optimalt datagrunnlag for en kvantitativ test) der vurderingskriteriene bunner i hvor tilgjengelig løsningen er for det allmenne marked, hvor forståelig publikasjonene som presenterer er for rapportens forfatter og popularitet i industrien - det er jo tross alt et problem av industriell undertone som besvares her.

Kapittel 4 kalles for \emph{Levende oppgradering av datamodeller realisert med Project Voldemort}. Dette kapitlet vil presentere systemdesignet rundt en kommersiell webapplikasjon kalt ''DBUpgradinator'', som kan inndeles i en frontend-del med presentasjonslogikk formet av CSS - og JS - filer og en backend-del med kontroll-logikk skrevet i Java som ved hjelp av serialisering med binærdatakodingsbiblioteket Apache Avro snakker med datalageret, en instans av den distribuerte oppslagstabellen Voldemort. I kapittel 4 blir også en kontinuerlig oppdateringsleveranseløsning for hele applikasjonen, inklusive dets implisitte dataskjema, skissert.

Kapittel 5 beskriver hvordan måloppnåelse av tidligere beskrevede krav til datamodellevolusjonsverktøyet evalueres. Under testing vil webapplikasjonen DBUpgradinator gjennomgå en patch der den implisitte datamodellen sett fra applikasjonslagets perspektiv blir endret samtidig som en mengde genererte forespørsler tilsendes tjenerne slik at en vanlig arbeidslast med spørringer simuleres.

I kapittel 6 konkluderes evalueringen og i det beskrives forslag til videre kvantitativt feltarbeid som kan bygge på denne analysen.
