\clearpage
\pagenumbering{roman} 				
\setcounter{page}{1}

\pagestyle{fancy}
\fancyhf{}
\renewcommand{\chaptermark}[1]{\markboth{\chaptername\ \thechapter.\ #1}{}}
\renewcommand{\sectionmark}[1]{\markright{\thesection\ #1}}
\renewcommand{\headrulewidth}{0.1ex}
\renewcommand{\footrulewidth}{0.1ex}
\fancyfoot[LE,RO]{\thepage}
\fancypagestyle{plain}{\fancyhf{}\fancyfoot[LE,RO]{\thepage}\renewcommand{\headrulewidth}{0ex}}

\section*{\Huge Sammendrag}
\addcontentsline{toc}{chapter}{Sammendrag}	
$\\[0.5cm]$

Moderne kommersielle programvaresystemer leverer ofte tjenester til opptil flere hundre tusen brukere over Internett, det vil si ved hjelp av HTTP - applikasjonsprotokollen. Det er på slike systemer at NoSQL - databaser gjerne tas i bruk da de i vesentlig grad er i stand til å lagre stadig større datavolum som oppstår i et stadig raskere tempo mer effektivt. NoSQL - databaser er også mer horisontalt skalerbare enn de tradisjonelle relasjonsdatabasene, det vil si at de egner seg bedre til å dele ut og kopiere dataelementer over en klynge av databaseprosesser. \\

En av de største utfordringene innen drift av moderne kommersielle programvaresystemer som bankapplikasjoner, sosiale medier og netthandelssytemer er kunsten å minimalisere nedetid som følger av oppdatering av systemet. For mange store bedrifter som eier og admistrerer slike systemer er det totalt uaktuelt å dekommisjonere hele eller deler av systemet for å installere en liten programvareoppdatering eller resirkulere minne. Til det vil nedetiden til systemet medføre utålelige inntektstap. Derfor oppgraderer mange bedrifter systemene sine ''online'', det vil si at opppgraderingen gjøres uten å slå av en eneste datamaskin, og uten å forstyrre behandlingen av forespørsler fra brukere. Erfaringer fra industriene tilsier at slike levende oppgraderinger er lettere sagt enn gjort, især når det kommer til oppgraderinger av applikasjonens datamodell mens den opererer i et produksjonsmiljø. \\

Moderne programvaresystemer utvikles gjerne under en smidig utviklingssykel der nye versjoner, eller oppdateringer, publiseres til bruk opptil flere ganger om dagen. Slike oppdateringer kan endre programvaresystemets datamodell, eller ''skjema'' som det heter i relasjonelle databaser. For å utføre slike opppgraderinger tryggest mulig blir systemet oppgradert på rullerende vis. Denne masteroppgaven setter som mål å realisere støtte for levende oppgradering av data-modeller i høytilgjengelige systemer uten nedetid ved å utvikle et eget administrasjonsverktøy til databasehandteringssystemet Voldemort. Dette verktøyet tillater applikasjonsutviklere å legge inn transformasjonsfunksjoner som kalles på ''lazy'' vis når hver enkelt datatuppel aksesseres i databasen. \\

\clearpage
