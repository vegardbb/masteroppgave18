%===================================== KAPITTEL 5 =================================

\chapter{Evaluering av datamodellevolusjonsløsning}

Dette kapitlet presenterer og diskuterer resultatene av testingen av datamodellevolusjonsløsningen DBUpgradinator, og sammenlikner datamodellevolusjonsløsningens kvaliteter med øvrige systemer presentert i kapittel 3. I delkapittel \ref{prog} vil testprosessen beskrives, herunder inkludert konfigurasjon av Voldemort-klyngeinstansen; hvordan testprogrammet WebshopSimulator er bygd opp; innsamling av testdata ved hjelp av loggeprogrammet Log4j og diverse tellere innebygd i simulasjonsprogrammet.

\section{WebshopSimulator} \label{prog}

Webapplikasjonen som DBUpgradinator testes på, heter \emph{WebshopSimulator}.

Frontend-delen av webapplikasjonen er skrevet i Javascript, og kjøres som et separat klientprogram. Dette skriptet kjøres i kjøretidsmiljøet NodeJS, slik at  , kalt \emph{WebAppSimulator}, simulerer en kontinuerlig serie med forespørsler som jevnfordeles blant applikasjonsinstansene. Programmets funksjonalitet kan i korte trekk inndeles i tre satser:

\begin{enumerate}
  \item Programmet sender klyngen av tjenere en serie av POST-forespørsler, slik at nye aggregater lagres på hver av de fire tomme nodene i databasen. Totalt 8 GB i serialiserte JSON-objekter sendes applikasjonsinstansene og Voldemort-klientene. Samtidig holder en egen loggeprosess styr på ID-verdiene som genereres av tjenerne som tilsendes denne simulerte klienten
  \item Etter at den første satsen er ferdig, terminerer programmet. Da må applikasjonsloggen, generert av frontendsimulatoren, transformeres til en liste av unike IDer. Til å oppnå dette formålet brukes tekstverktøyet Vim.
\end{enumerate}

\cleardoublepage

