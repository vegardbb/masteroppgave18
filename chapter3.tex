%===================================== KAPITTEL 3 =================================

\chapter{Relatert arbeid}

I dette kapitlet presenteres eksisterende moduler og programvare som kan brukes av programvareutviklere til å migrere data fra èn datamodell til en annen under appliaksjonsoppgradering. Kapitlet vil begynne med å beskrive tilgjengelighet som kvalitetsattributt i moderne webapplikasjoner. Det vil også skissere viktigheten av  levende eller rullerende oppgradering i lys av kontinuerlig leveranse. I tre påfølgende delkapitler beskrives tre forskjellige oppgraderingsrammeverk: Utgavebasert oppgradering av datamodeller, kalt EBR; KVolve, en datamigrasjonsmodul integrert i nøkkelverdisystemet Redis; Imago, et atomisk rammeverk for helhetlig webappliaksjonsoppgradering. Et siste relaterte arbeid som beskrives er en datamigrasjonsmodul for Apache Cassandra som er implementert i programmeringsspråket Java, kalt \textbf{cassandra-migration-tool-java}.

\section{Høytilgjengelige webapplikasjoner}

Tilgjengelighet er i mange moderne web 2.0 - applikasjoner en kritisk kvalitet som det stilles svært strenge krav til, især systemer som brukes av flere hundre tusen mennesker verden over. Slike systemer stilles derfor høye krav til tilgjengelighet, og refereres i litteraturen til som høytilgjengelige systemer (eng. ''high-availability''). For å tilfredsstille høye tilgjengelighetskrav, hvis kvantifiserte prosentandel i en tjenestenivåavtale gjerne er basert på formelen for steady-state-tilgjengelighet, må man forstå hvilke typer feil som oppstår, hvorfor de oppstår, og i hvilke omstendigheter de oppstår. Dernest kan man implementere strategier for å redusere feilenes påvirkning på programvaren. Høytilgjengelige systemer lover som regel et pålitelighetsnivå på 99.9 \%, det vil si inntil 8 timer nedetid om året.

\subsection{Tilgjengelighet}

Et systems tilgjengelighet er kvalitetsattributtet som beskriver hvordan systemet oppfører seg i et abnormalt operasjonsmiljø, for eksempel ved et feilscenario eller et kontrollert avbrudd. Tilgjengelighet bygger på et annet kvalitetsattributt, pålitelighet, ved å legge til reparasjonsaspektet - når et system utsettes for en feil, så vil systemet respondere på hendelsen slik at feilen ikke forårsaker nedetid \citep{BCK2013}. Kvaliteten beskriver altså mer enn bare hvor ofte systemet er tilgjengelig for å behandle tjenesteforespørsler fra dets brukere.

\cite{BCK2013} definerer et programvaresystem sin tilgjengelighet som dets evne til å maskere eller reparere feil slik at dets sum av alle perioder med nedetid innenfor et definert tidsintervall i forhold til nevnte intervall ikke overstiger en bestemt prosentandel. Med ''nedetid'' menes en konkret, lukket tidsperiode (fra og med ett tidspunkt til og med et annet) der systemet ikke er mottakelig for kommando fra dets påtenkte brukere.

Mer presist beskrevet er steady-state-tilgjengelihet, en indikator på oppetiden (det motsatte av nedetid) til en komponent av et system eller systemet sett under ett, bestemt av to estimerte forventningsverdier: Tiden det tar før en svikt oppstår (MTBF), og reparasjonstiden (MTTR). Steady-state-tilgjengelihet er et forholdstall uttrykt matematisk som $ \frac{MTBF}{MTTR+MTBF} $.

I kontekst av programvareutvikling bør denne formelen tolkes dithen at for å diskutere systemets tilgjengelighet må sannsynlige scenarier der feil oppstår og hvordan handtere dem identifiseres. Effektene av hver feil må estimeres. Ikke minst må man tenke hvor lang tid man må bruke på å reparere eller maskere den enkelte feil. I tillegg kan feil unngås og fjernes, eventuelt kan man velge ikke å gjøre noe som helst i det de oppstår \citep{BCK2013}.

For å øke et systems tilgjengelighet må man minimalisere dets samlede nedetid ved å minimalisere effekten feil har på det. Feil (eng. ''fault'') forårsaker svikt i systemet (eng- ''failure''), skriver \cite{BCK2013}. En svikt er et avvik i systemets oppførsel fra dets spesifikasjoner. Implisitt i denne forklaringen av begrepet ligger at systemsvikt er synlig for en menneskelig observatør, si en autorisert bruker til eksempel. Eksistensen til en systemsvikt er altså ikke definert før den observeres.

Til vanlig telles ikke planlagt nedetid, som oppstår i forbindelse med vedlikeholdsoperasjoner som programvareoppgradering eller generell minnerensing ved restart av tjenere, i beregningen av tilgjengelighetsprosenten som typisk baseres på den tidligere nevnte formel. Vedlikehold utføres gjerne på et planlagt tidspunkt der forespørselstrafikken er på et svært lavt nivå. Forskningslitteratur dette dokument henviser til indikerer at planlagt vedlikeholdsarbeid på et kjørende, distribuert system utgjør en egen klasse av mange forskjellige feilkilder som kan medføre nedetid, dog tradisjonelle feiltoleransemekanismer fokuserer på å motvirke, unngå, eller tillate uventede feil \citep{dumitracs2009upgrades}.

%Til tross for at denne detaljen opplyses i tjenestenivåavtalen der programvaresystemet er totalt utilgjengelig grunnet vedlikeholdsarbeid, men diverse brukere forespør dets tjenester likevel.

% TODO: BASE og PACELC

\subsection{Versjonskappløp i distribuerte systemer}

% Mixed version race
\citep{dumitras2010upgrade} stiller følgende hypotetiske scenario om risiko forbundet med rullerende oppgradering av distribuerte kjøringsmiljøer: I en webbasert nettbankappliaksjon som komunniserer med et banksystem bestående av mange tjenere oppdages en potensielt farlig anomalie sikkerhetsmessig. I den gamle versjonen av klientprogramvaren finnes en tekstboks der brukeren skriver inn informasjon om pengeoverføring, både antall pengeenheter så vel som valutaen enhetene måles. Ergo aksepterer denne tekstboksen alle alfanumeriske tegn, hvilket betyr at nettbanken står i fare for å bli utsatt for SQL-innsprøyting gjennom web-grensesnittet.

For å motvirke denne sikkerhetsrisikoen beslutter ingeniørne som vedlikeholder programmet at den enkelte bruker skal velge pengevalutaen med en radio-boks istedet for å skrive den inn på tekstform. Dermed vil tesktboksen kun måtte akseptere tall. Imidlertid avdekkes en annen risiko under integrasjonstesting av denne feilreparasjonen: Denne endringen i klientdelen av arkitekturen medfører at tjenerapplikasjonen nå mottar to seperate parametre istedet for den ene strengen fra tekstboksen, og mens tjenerapplikasjonen oppgraderes, vil tjenere som kjører den utdaterte applikasjonen kun lese innholdet fra tekstboksen, som i den nye versjonen klientprogrammet bare er et tall, og følgelig anta på grunn av oppførselskonfigurasjon at beløpet som sendes er oppnevnt i amerikanske dollar. Denne inkonsistensen under oppgradering kan skape betydelig trøbbel for kunder som skal overføre penger i ikke-amerikanske valutaer.

Slike scenarier omtaler \citep{dumitras2010upgrade} som versjonskappløpsbetingelser ''mixed version race''. I hans artikkel ''To Upgrade Or Not To Upgrade'' demonstreres en sannsynlighetsmodell for risikovurdering av rullerende programvareoppgradering med spesielt hensyn på versjonskappløpsbetingelser, tilstanden systemet har der responsen på en forespørret operasjon er forskjellig mellom to noder som kjører hver sin versjon av nodeprogramvareprosessen.

\subsection{Dynamisk programvareoppdatering (DSU)}

\cite{hicks2005dsu} introduserer et generelt, fleksibelt, og effektivt oppdateringsrammeverk for å oppdatere en kjørende programvareprosess, kalt \emph{Dynamic Software Upgrade}. Artikkelen er en utvidet versjon fra et foredrag Michael Hicks holdt for ACM i 2001.

Én ofte benyttet tilnærming for å innføre online oppgradering er å installere oppdaterte versjoner av programmet på reserve-tjenere som kan settes ut i produksjon umiddelbart hvis den aktive produksjonstjeneren utsettes for et feilscenario \citep{hicks2005dsu}.

For eksempel benyttet Visa tidlig på 2000-tallet hele 21 store datamaskiner til å kjøre sitt transaksjonsprosesseringssystem (TPS) på. Ved oppgradering ble den enkelte instansen sin tilstandsdata bevart mens de enkelte prosessene ble byttet ut. Transaksjonsprosesseringssystemet til Visa opererte per 2001, ifølge \cite{hicks2005dsu}, med et oppetidkrav på 99.5 \%, men gjennomgikk tusenvis av mindre oppdateringer per år. Denne maskinbundete oppgraderingsteknikken er nødvendigvis meget dyr og gjør utviklingsprosessen for de som vedlikeholder programvare ekstra tungvint. Dette grunnet den nødvendige rutinen med å overføre tilstandsdata fra en gammel applikasjonstjener til en ny og nødvendigheten av å kjøpe inn og starte opp en ny maskin hver gang en oppdatering skal installeres \citep{hicks2005dsu}.

DSU – rammeverket omgår de ekstra oppgraderingskostnadene skissert i avsnittet ovenfor. \cite{hicks2005dsu} beviser at det innehar fire forskjellige kvaliteter: Fleksibilitet (en hvilken som helst komponent i systemet kan oppdateres uten at hele systemet må slåes av); robusthet (systemet minimaliserer risikoen for systemfeil ved oppdatering); brukbarhet (oppgraderingsprosessen er enkel å forstå); ytelse (oppgraderingsprosessen skal ikke utføres på bekostning av systemet tjenesteevne).

Gyldighetsområdet til DSU – Rammeverket er ment å være programvaren på én enkelt maskin, det kan med andre ord ikke brukes på for eksempel for distribuerte databaseinstanser tilhørende en webapplikasjon. Endringer i lagringsstrukturen påkrever en koordinert oppdatering av alle applikasjonsinstanser som kommuniserer mot den kjørende databasens grensesnitt.

Systemer som oppdaterer dynamisk bruker litt tid på starte opp en ny prosessnode og å overføre data fra den gamle til den nye, men kan samlet sett redusere oppgraderingstiden vis-a-vis distribuerte systemer som implementerer online-oppgradering ved hjelp av varme reserver, det vil si en annen fysisk maskin som overtar arbeidet fra tjeneren med den gamle applikasjonen.

En dynamisk oppdatering gjennomføres ved å starte den nye overtakende prosessen, fryse tilstanden til det opprinnelige programmet og overføre tilstandsdata over til den nye, blanke prosessen, og til sist fase ut den gamle til fordel for den nye. Det mest sentrale elementet i denne prosessen er en rutine for å sammenlikne semantiske forskjeller mellom de to versjonene og for å finne ut hvordan oppførselen til den nye versjonen er forskjellig fra den gamle.

Ved en dynamisk oppdatering må applikasjonsdata i den gamle, kjørende versjonen overføres til den nye, innkommende versjonen overføre defineres en tilstandstransformasjonsfunksjon som kan transponere en tilstandsvariabel fra den gamle versjonen til en tilsvarende verdi for den nye.

\cite{hicks2005dsu} omtaler to forskjellige måter å implementere dynamisk oppdatering, eller ''dynamic patching'', som det også kalles. Den første innebærer å først kompilere den nye versjonen fra den endrede kildekoden, instansiere den nye versjonen i en egen prosess på lik linje med prosessen til den gamle versjonen av programmet, og deretter beordre sistnevnte til å overføre sine tilstandsdata over til prosessen som kjører den nye versjonen. Etterpå bruker den nye versjonen tidligere nevnte transformeringsrutine for å konvertere tilsendte tilstandsdata. Så vil den nye prosessen begynne sin kjøretid med de transformerte dataene fra tidligere.

Den andre måten går ut på å lenke oppdateringen inn i den eksiterende, kjørende prosessen, transformere tilstandsdata der de er (her blir tilstanden kopiert til en annen adresse i prossessens adresserom i minne) og deretter endre koden prosessen kjører fra direkte ved hjelp av lenken.

\subsection{Nedetid forårsaket av vedlikeholdsarbeid}

En undersøkelse fra 1998 av totalt 426 høytilgjengelige programvaresystemer avdekket 75 \% av totalt 6000 tilfeller av systemnedetid kom av planlagt vedlikeholdsarbeid på enten maskinvare eller programvare, og disse planlagte nedetidsperiodene varte som regel dobbelt så lenge som de periodene med nedetid som kom av uventet systemsvikt \citep{lowell2004, dumitras2009nodowntime}. Feiltoleransemekanismer lages utelukkende for å kunne forhindre sistnevnte type nedetid. Dette har en sammenheng med at planlagt nedetid tradisjonelt ikke inngår i beregningen av total systemtilgjengelighet per år, fordi systemvedlikehold kunne gjøres på tidspunkt der systemets tjenester ikke etterspørres (for eksempel om natten). Høytilgjengelige systemer påkreves gjerne å være tilgjengelig 24 timer i døgnet, så denne betingelsen er naturligvis ikke lengre gjeldende.

Oppgradering av store informasjonssystemer er også svært dyre, grunnet større endringer i dataskjemaet/dataformatet og/eller datamigrasjon. Oppgraderinger som innebærer komplekse endringer i datamodellen medfører at dataene må gjennomgå tunge og langtekkelige konverteringer som kan ta flerfoldige timer for å konformere til nevnte endringer. Disse oppgraderingene er vanskelig å installere mens systemet leverer brukerne sine tjenester uten å forstyrre tjenesteleveransen såpass mye at systemets ytelse faller ned på uakseptable nivåer. Dette betyr at tunge oppdateringer i datamodellen ofte må gjøres offline, når systemet er slått av. Som en konsekvens av dette unnlater systemadministratorer i det lengste å endre datamodellen når applikasjonen er i et kjørende produksjonsmiljø \citep{dumitras2009nodowntime}.

\cite{dumitras2009nodowntime} hevder at manuelt styrte oppgraderingsmetoder som fordrer mye kunnskap om applikasjonslagets interaksjoner med datamodellen ikke er veien å gå for å nærme seg idealet om tilnærmet døgnåpen tjenesteytelse. Med sitt helhetlige system for gjennomføring av oppgraderinger med komplekse datamodellendringer, Imago, ønsker \cite{dumitras2009nodowntime} også å unngå behovet for å holde styr på avhengighetene til den kjørende/gamle versjonen. Det er jo tross alt et NP-hardt problem.

Ved å gjennomgå oppgraderingshistorikken til Wikipedia (\url{http://www.wikipedia.org}) påviser \cite{dumitras2009nodowntime} de viktigste årsakene til at nedetid blir planlagt ved systemoppgradering. Wikipedia er den mest ettertraktede kilden til informasjon på Internett. Leksikonet driftes på wiki-plattformen MediaWiki, implementert i PHP, som aksesserer en distribuert infrastruktur bestående av flere hundre databasetjenere.

\cite{dumitras2009nodowntime} stiller følgende eksempel for en skjemaendring (i en relasjonell database), illustrert i figur 2 på artikkelens tredje side: Kolonne \texttt{a} i tabell \texttt{X} blir byttet ut med kolonne \texttt{b}, begge kolonner er også representert i tabell \texttt{Y} hvis attributtverdier brukes til å initialisere attributtverdiene for \texttt{b} i tabell \texttt{X} med joinbetingelsen \texttt{Y.a = X.a}. Når kommandoen \texttt{DROP COLUMN a} kalles under \texttt{ALTER TABLE X}, vil påfølgende spørringer fra uvitende tjenere som kjører den gamle versjonen av MediaWiki påkalle databasefeil. På motsatt vis går det ikke an for oppgraderte MediaWiki-tjenere å spørre databasetjenere som kjører den gamle datamodellen der kolonne \texttt{b} ikke er definert i tabell \texttt{X}. Dermed må systemoppgraderingen foregå i to atomiske steg, der samtidige spørringer trygt kan komme innimellom: Først defineres kolonnen \texttt{X.b} og initialiseres med data fra \texttt{Y.b} ved hjelp av tidligere nevnte joinbetingelse. I samme steg oppgraderes applikasjonstjenerne til MediaWiki (dvs mengden av PHP-skript). Før dette steget er ferdig gjennomført får ikke klienter lov til å sende forespørsler overhodet. Steg 2: Slett kolonnen \texttt{X.a}.

Ved online oppgradering er det meningen at klienter kan aksessere systemet samtidig som det oppdateres. Den automatiserte prosedyren må derfor ta høyde for innkommende \texttt{UPDATE}, \texttt{INSERT} og \texttt{DELETE} - spørringer og samkjøre disse forespurte endringene med kolonnetillegget i tabell \texttt{X}. I praksis betyr dette at resultatrelasjonen fra joinbetingelsen blir kalkulert på nytt for hver nye forespørsel som inntreffer. \cite{dumitras2009nodowntime} argumenterer at sjonglering med versjonmiks mellom to forskjellige lag av systemet kan fort medføre ekstra ytelsesoverhead iform av forlenget oppgraderingstid sett i forhold til hvis den ble utført offline.

\cite{dumitras2009nodowntime} sin studie av MediaWiki sin oppgraderingshistorikk kan oppsummeres såeldes: Inkompatible skjemaendringer forhindrer rullerende oppgraderinger og påbyr at oppgraderingen skjer i ett helhetlig steg. Dataavhengigheter er vanskelige å synkronisere med endringer som følger av innkommende forespørsler og kan påføre ekstra prosesseringstid for oppdateringsprosessen samt overbelastning av lagringsressurser under datakonvertering og/ellere migrasjon. Dataskjemaendringer som motstrider hverandre, lik som flettekonflikter i asynkrone distribuerte databaser, krever manuell intervensjon.

% Revisjon: Er dette sant? I distribuerte miljøer, kan man lite på databasereplikering når man oppgraderer rullerende?
Når programvaren, eventuelt operativsystemet som kjører på hver enkelt datamaskin i et produksjonsmiljø spredt utover opptil flere datasentre på vidt forskjellige geografiske lokasjoner skal oppgraderes fra én versjon til en nyere, så har systemets administratorer flere valg. Man kan slå av alle noder i hele systemet, eventuelt hele datasentre om gangen, for deretter å installere oppdateringen på én og én node. I systemer med et moderat antall datamaskiner som tilsammen leverer en applikasjonstjeneste over et nettverk av klienter, og som alle er i ett og samme tjenerrom samlet i én klynge over et lokalt nettverk, vil denne såkalte ''stopp-verden'' - strategien, som \cite{saur2016} nevner, være en gangbar strategi som innebærer lave, tolererbare kostnader i form av tapt tjenestetid for brukerne. Til gjengjeld er man garantert at alle de oppdaterte 

% \cite{dumitras2009nodowntime} refererer til flere andre undersøkelser av systemer der feil har oppstått under systemoppgradering eller øvrig vedlikeholdsarbeid.


% Kapittel 3.2 - Kontinuerlig leveranse
\section{Kontinuerlig leveranse}

Blogginnlegget til \cite{hauer2015} handler om behandling av databaser i en applikasjonlivssyklus der utviklingsmetodikken ''Kontinuerlig Leveranse'' (eng. ''Continous Delivery'') gjennomføres. For hver gang ny kode skal kjøres på applikasjonstjenerne må databasksjemaet og følgelig eksisterende data oppdateres på samtlige databaseinstanser før applikasjonsoppdateringen kan selv tre i kraft.

I webapplikasjoner med mange brukere opparbeides mye lagret data over tid. Følgelig blir datamigrasjoner som følge av endringer av applikasjonslogikken også mer tidkrevende. Hvis det stilles høye tilgjengelighetskrav til webapplikasjonen kan ikke denne migrasjonsprosessen utføres i ett, langtekkelig steg. Når denne migrasjonsprosessen er ferdig utført er det minst like tidkrevende å ''rulle tilbake'' disse endringene, og det lar seg nødvendigvis ikke alltid gjøre. Dette er særlig tilfellet i produksjonsmiljø der databasesystemet integrerer data fra flere separate uavhengige applikasjoner, for eksempel et intranett for et firma med ett eneste databasesystem som handterer data både for salgsavdelingen og varebeholdningsavdelingen. Således anbefaler \cite{hauer2015} å separere datakilder i en mikrotjenestearkitektur og derfor benytte et persisteringsmønster \cite{sadalage2013} refererer til som ''polyglot persistance''.

I et kontinuerlig leveransemiljø med relasjonelle datamodeller anbefaler \cite{hauer2015} å gjøre skjemaendringer ''backwards compatible'', altså at den nye versjonen av dataskjemaet kan brukes av applikasjonslogikken som er koplet til det gamle databaseskjemaet. Et eksempel er fjerning av et attributt fra en tuppel. Denne endringen er ikke direkte bakoverkompatibel. Derfor må denne endringen oppdeles i et sett med endringer: Først må attributtet defineres til ikke å ha en påkrevd verdi, på fagspråket kalt ''nullable''. Dernest må applikasjonslogikken på hver appliaksjonstjener oppdateres slik at attributtet ikke etterspørres. Til sist kan det ubrukte attributtet fjernes fra databaseskjemaet.

\subsection{Skjemaendringer og datamigrasjon i NoSQL-databaser}

NoSQL-datamodeller har til felles at de ikke eksplisitt deklarerer et databaseskjema slik som man ser i Postgres eller MySQL. Ettersom ethvert aggregat kan i prinsippet ha en hvilken som helst objekt-struktur der egne nøstede lister og objekter og hvilke som helst variable er inneholdt. Kort sagt, i et nøkkelverdi kan man lagre det man vil per nøkkel. Skjemaendringer er ikke et vesentlig problem i et kontinuerlig leveranse-utviklingsmiljø. Når man ikke trenger å vite strukturen på dataobjektene som lagres på forhånd, blir kontinuerlig endring og publisering til produksjon av applikasjonen tilsynelatende enklere.

Som \cite{sadalage2013} poengterer at enhver applikasjon som persisterer sine tilstand gjør til enhver tid antakelser på hvordan aggregatet den får tak i per oppslagsnøkler ser ut. Applikasjonens kildekode gjengir et implisitt skjema. Hvis koden er godt nok strukturert, for eksempel ved å definere egne modell-klasser som eksplifiserer applikasjonens dataskjema, vil jobben med å identifisere endringer i applikasjons usynlige dataskjema lettere.

Å migrere data i et høytilgjengelig produksjonsmiljø er dessverre ikke stort lettere med NoSQL enn i relasjonsdatabaser. Migrasjon er en aktivitet som vil nødvendigvis forekomme mye oftere i en kontinuerlig leveranse-livssyklus, og er desto mer tidkrevende jo mer omfattende endringen er. \cite{hauer2015} demonstrerer to endringsscenarier der levende datamigrasjon må utføres:

\begin{description}
  \item [Enkel endring i datamodellen] Et eksempel på en enkel datamodellendring er å endre datatypen til attributtet ''postNummer'' i \texttt{Kunde} - aggregatet fra heltall til streng, jamfør modellen i \ref{fig2}. Så lenge applikasjonen klarer å handtere den eksisterende datatypen for attributtet så vel som den nye så trenger vi ikke å oppdatere data som allerede er lagrert i databasesystemet, med mindre de absolutt \textbf{må} være konsistente med den nye applikasjonsversjonen.
  \item [Kompleks endring i datamodellen] Herunder inkluderes omarrangering av strukturen til aggregatet applikasjonen bruker, som endring av objektnøstingen oppsplitting eller fletting av aggregater. Jamfør \ref{fig2}, et eksempel på en slik kompleks endring kunne være å tillate en kunde å registrere flere leveringsadresser istedet for bare én.
\end{description}

Den sistnevnte endringen er ofte en nødvendig handling for å ''endre'' spørringer til databasen. I prinsippet kan man ikke ''endre'' spørringer i NoSQL-modellen, det finnes bare én type lesespørring og én skrivespørring. Følgelig blir programvareutviklere nødt til å strukturere aggregatet til å passe med de funksjonelle krav som stilles til spørringer i databasen. Når de funksjonelle kravene til applikasjoner hvis datamodeller er aggregatbaserte endres, så vil krav til spørringsresultater endres deretter og følgelig må datastrukturen til aggregatet som lagres også endres. Den logiske konklusjonen, som \cite{hauer2015} påpeker, er at omfattende datamigrasjoner gjøres oftere hvis datamodellen er enten delvis strukturert eller totalt ustrukturert, enn i strengt strukturerte datamodeller.

Der semistrukturerte datamodeller har stor \underline{skjemafleksibilitet} har strengt strukturerte datamodeller stor \underline{spørringsfleksibilitet} (gitt at de er normaliserte). I SQL-databaser trenger man ikke endre databaseskjemaet, metadataene som beskriver relasjonene i databasen, for hver bidige gang de funksjonelle kravene endres. Grunnen til det er at spørringene kan endres ved hjelp av \texttt{SELECT} og \texttt{JOIN} - operandene, i tillegg til aggregeringsoperander tilbudt av spørringsgrensesnittet.

% Hvordan NoSQL-modeller kan nyttes ved oppgradering av distribuerte systemer
Semistrukturerte datamodeller, en iboende egenskap i de aggregatorienterte modellene \cite{sadalage2013} presenterer, gir den som er interessert i rullerende oppgradering av distribuerte webapplikasjoner en svært fleksibel vei hva angår dataskjema-oppgradering. Sett fra databasetjenesten sitt perspektiv er det ikke noe i veien for at strukturen i ''verdiene'' tilknyttet nøklene i datalageret ikke er samstemte seg imellom, i motsetning til den rigide ordningen av tupler i relasjonsdatabaser som MariaDB og PostgresSQL. Det er nemlig ikke alle typer skjemaendringer som lar seg utføres i databasetjenere på rullerende vis. Et gjenstående problem med rullerende applikasjonsoppdatering er, uansett hvor strukturert applikasjonens datamodell er, at mens node etter node skiftes ut til en ny instans vil instanser av den oppgraderte applikasjonstjeneren generelt sett respondere på brukerforespørsler på en annerledes måte enn instanser av den gamle versjonen som ennå ikke er oppgradert. Dette problemet er kjent under navnet ''Mixed Version Race'', som er et gjennomgående tema i \cite{dumitras2010upgrade,dumitracs2009upgrades}.

I diskusjonen om hvordan semistrukturerte datamodeller kan oppgraderes uten å avbryte tjenesteleveranse i produksjonsmiljøet, er visse fakta om aggregatorienterte datamodeller og NoSQL - modeller forøvrig, relevante å komme i hu. Slike databaser har ingen semantisk formening om data som lagres, de bare lagres på en eller flere lagerinstanser i et distribuert system. Selve meningen til de dataobjektene handteres og utvikles i selve applikasjonen sammen med dets kildekode. I datalagerdelen av systemets arkitektur er disse objektene ofte representert som JSON-strenger, ProtoBuf-objekter, Avro-objekter og så videre.

%  PACELC
%  BASE

% Kjelde: https://github.com/smartcat-labs/cassandra-migration-tool-java
\subsection{Migrasjonsverktøy for Cassandra}

I migrasjonsverktøyet \emph{cassandra-migration-tool-java}, som er et tredjepartsverktøy til Apache Cassandra for databaseadministratorer, gjøres versjoner av det implisitte skjemaet til en applikasjon eksplisitt, og lagres i en separat metadatatabell, kalt \texttt{schema\textunderscore version}, i form av en identifikatorstreng og et tidsstempel \citep{bozic2015}. Verktøyet, som er åpent tilgjengelig på GitHub, kan utføre to forskjellige migrasjoner: Skjema-migrasjoner, der databaseskjemaet (tillegg og fjerning av attributter, tillegg av ''tabell'') endres, og datamigrasjoner (endringer av eksisterende aggregat i databasen) \citep{bozic2015}. Ved skjemamigrasjon er det, iallfall for modulens forfattere, kritisk for programmet å sikre oppdateringskonsistens i databaseklyngen hvis databaseadministratoren har behov for det. 

Derfor er en konsensusprotokoll implementert der alle spørrende klienter nødvendigvis må aktivt vente på at alle databasenoder har migrert ferdige sine egne skjema før de kan sende lese - og skriveforespørsler. Konsekvensen av ikke å føre en slik konsistenskontroll på replikanivå er at applikasjonsklientene eksponeres for versjonmiksen i datalageret, noe som åpner for versjonskappløp replikaeneimellom per nøkkel ved både lese - og skriveoperasjoner.

\section{EBR}

Online oppgradering av distribuerte programvaresystemer blir stadig mer nødvendig for stadig flere vriksomheter. Å planlegge nedetid for hele systemer,selv om det er for en god sak, nemlig viktige vedlikeholdsoperasjoner, er logisk ekvivalent med å avbryte flyten i forretningsprosessene og følgelig tape inntekter og kundetilfredshet, med vilje vil kanskje utenforstående si. Behovet for online oppgraderinger er definitivt prekært, og en suksessful implementasjon av automatisert oppgradering vil utvilsomt gi sterk datadrevne tertiære bedrifter et viktig konkurransefortrinn \citep{choi2009}.

Utgavebasert redefinisjon (EBR) er en oppgraderingsmetode ment for å oppgradere databaseprogrammet som driver datalaget til en applikasjon, uten å forårsake nedetid for applikasjonens sluttbrukere. Denne metoden må applikasjonens utviklere ha god kjennskap til når de implementerer selve oppdateringsskriptet, ellers er det vanskelig å få oppgradert applikasjonen online uten å påføre systemsvikt \citep{choi2009}.

EBR har en vært fast funksjonalitet i Oracle - databasesystemet siden versjon 11gR2, og ser ut til å være ganske populær blant store aktører med produksjonsmiljø som aksesseres av mange brukere verden over 24 timer i døgnet. \cite{choi2009} kommenterer at både BetFair og IFS, to store tertiære selskaper, er ivrige brukere av patching med EBR. Disse to kundene av Oracle demonstrerer at det er et meget stort marked for nedetidsfri programvaresystemoppdatering. Et kjapt internettsøk avslører at denne oppgraderingsmetoden er fremdeles ønsket og relevant for datadrevne applikasjoner som benytter en relasjonell datamodell. \footnote{En video der en ivrig EBR-bruker forklarer metodens fordeler er å finne på \url{https://www.youtube.com/watch?v=nRiAQgNDgoA}}

\subsection{Løsning}
I versjon 11.2 av Oracle definerer \cite{choi2009} til i alt tre nye konsepter for å muliggjøre online oppgradering:

\begin{itemize}
  \item Utgave (Edition). Hver instans av for eksempel en funksjon, SQL-setning, pakke eller view (immuterbar dataobjekt) eksisterer som en egen entitet, en utgave av nevnte konstruksjon. Således definerer konseptet om utgaver en måte å isolere variasjoner av det samme kodeobjektet fra hverandre, slik at man kan operere på en spesifikk instans av gangen
  \item Utgaveperspektiv (Editioning View). Gjennom bruk av projeksjoner gjemmer utgaveperspektivet nyopprettede attributter eldre utgaver av koden ikke er ment å se fordi attributtet ikke skal eksistere i deres verdensbilde
  \item Triggere på tvers av utgaver (Cross Edition Triggers). For å synkronisere data forskjellige utgaver ikke har tilfelles brukes triggere. Dataendringer som inntreffer hos tupler tilhørende eldre utgaver propageres til kolonnene nye utgaver leser ifra, og vice versa
\end{itemize}


Meningen med utgaver er å sette opp et isolert miljø slik at mengden av kodeobjekter (instanser fra kildekoden) som trygt kan endres i tandem gjøres således. Å inkludere dataverdier i seg selv i dette miljøet vil åpenbart medføre mange unødige, tunge dataoperasjoner i den grad datatupler dermed må oppdateres og kopieres fram og tilbake for hver opp - eller nedgradering. Derfor behandles kolonner og tabeller som statiske elementer, åpent synlig for alle versjoner. Således blir deling av data enklere. Semantisk sett må det gå raskt å opprette nye utgaver, men samtidig må tilstanden til koden pre-patch også være tilgjengelig for nye utgaver \citep{choi2009}. Utgaver innehar derfor følgende semantikk:

\begin{enumerate}
  \item Det finnes to typer objekter: versjonerbare (editionable) og ikke-versjonerbare, hvorav sistnevnte har den statiske egenskap at de er konstant identiske og synlige for alle utgaver (herunder inngår dataobjekter som tabeller og kolonner)
  \item Når en ny utgave opprettes vil den holde på en komplett historikk av alle versjonerbare objekter
  \item Endringer av versjonerbare objekter er kun synlige for versjonen der nevnte endring ble utført
\end{enumerate}

For å illustrere hvordan de tre ovennevnte konseptene fungerer i hop for å realisere en trygg oppgraderingsprosess som kan brukes til online patching, så vel som tilbakerulling av oppgraderinger, beskriver \cite{choi2009} et case med et eksempelskjema levert av Oracle Corp. I caset blir et attributt for telefonnummer splittet i to, en for landskode og en for telefonnnummer innad i landet med den korresponderende landskoden. I den datamodellen utvinnes to nye attributter og én ny funksjonell avhengighet fra landskode på telefonnummer. I denne patchen blir også en del SQL-prosedyrer oppdatert for å samsvare med denne attributtsplittingen. Omfanget av dette caset representerer en typisk patch for en datamodell som allerede kjører i et produksjonsmiljø.

Tilbakerulling av en oppgradering i tilfelle en svikt oppsto med den nye utgaven er såre enkelt:
\begin{enumerate}
  \item Post-patch - utgaven blir slettet
  \item Nye tabeller samt nye kolonner i eksisterende tabeller som ble opprettet i løpet av oppgraderingen blir gjemt bort takket være projeksjonene til utgaveperspektiv-komponenten
\end{enumerate}

Ubrukte tabeller og kolonner kan dermed "tas fram igjen" hvis man forsøker å installere patchen på nytt og lykkes. Tilbakerulling av oppgraderinger har ingen effekt på tilgjengeligheten til pre-patch - utgaven.

\subsection{Diskusjon}
\cite{choi2009} beskriver metoden utgavebasert redefinering (eng. Edition-Based Redefinition) som er implementert i Oracles database. EBR er en patchingmetode for databaseapplikasjon-instanser slik at patching av databasen ikke påkrever planlagt nedetid av appliaksjonssystemet i sin helhet. Vitnesbyrd fra to store firma beviser at EBR fyller et sårt behov for online programvareoppgradering. Dataobjekter, som for eksempel tabeller, er inkompatible med utgavemetoden. EBR fokuserer på prosedyrer og funksjoner databasesystemet er bygget med. Med tanke på at artikkelen ble utgitt 2009 er det tenkelig at de fleste industrier, deriblant telekom, allerede har anvendt manuell, rullerende oppgradering av sine distribuerte systemer over lang tid, og møtt på overraskende og dyre forviklinger som ikke kunne ha blitt forutsett uansett hvor erfarne databaseadministratorene er.

Som \cite{oliveira2006understanding} påpeker i deres oppsummering av undersøkelsene sine er det menneskelige feil som står for majoriteten av feil som oppstår i databaser som kjører i et produksjonsmiljø. Feil som databaseadministratoren forårsaker er som regel ikke maskerbare med for eksempel backup - teknikker eller redundansstrategier, og slike feil kan medføre blant annet at lagrede data blir fullstendig utilgjengelige, sikkerhetssårbarheter oppstår og at systemets ytelse svekkes betraktelig.

Et relevant spørsmål man kan stille denne oppgraderingsmetodologien er Hvordan handterer den miksede versjonstilstander? Det interessante svaret er at skopet til EBR er det relasjonelle databasesystemet applikasjonen persisterer sin datatilstand med. Versjonskappløp og miks av datalagversjoner og tjenestelagversjoner/controllerlagsversjoner er kun relevant for rammeverk som anskuer hele webapplikasjonsstakken og som versjonerer datamodellen separat fra selve applikasjonen, noe som faller naturlig for applikasjoner hvis datamodell opprettholdes av en relasjonell struktur.

Så kan man alltids lure på hvorfor \cite{choi2009} ikke stiller opp med en eksperimentell evaluering av teknologien de framstiller, all den tid designvalgene begrunnes blant annet med ensynet på ytelse. Kanskje oppleves skryten fra markedet eller det faktum at EBR fortsatt er en funksjonalitet av Oracle-databasen, i det vi skriver versjon 12c, som ble sluppet den 1. mars 2017, som bevis nok for bibliotekets utviklere. Dessuten er ordentlig testing av distribuerte systemer med ordentlige distribuerte arbeidslast en ganske dyr affære, og det er trolig grenser for hver presise, kostbare og omfattende tester Oracle tillater for et eneste aspekt ved Oracle RDBMS.
 % Trenger mer stoff, to delkapitler til
\section{Imago}

% Innledning
Her følger en presentasjon av en helhetlig systemarkitektur som unngår versjonsløpsbetingelser og gjennomfører atomiske, konsistente datakonverteringsprosedyrer med minimalt behov for manuell styring. Med dette designet foreligger en vesentlig økning både i maskinberegninger (CPU), og bruk av lagringsressurser (disk). \cite{dumitras2009nodowntime} påpeker at designet kan innføre flere oppgradingsscenarier uten å påtvinge planlagt nedetid for applikasjonen som kjører med dette systemdesignet.

\cite{dumitras2009nodowntime} undersøker årsakene til planlagt ved å se på oppgraderingshistorikken til Wikipedia. Endringer i skjemaet som krever at både applikasjonen og datalaget oppgraderes i ett enhetlig, atomisk steg fordrer vanligvis at hele programvaresystemet må slås av.

\subsection{Løsning}
Imago er bygget på tre grunnleggende målsetninger \cite{dumitracs2009upgrades}:
\begin{itemize}
  \item \textbf{Isolasjon}. Avhengigheter innad i det gamle unierset/versjonen av systemet må være adskilt fra oppgraderingsprosessen
  \item \textbf{Udelbarhet}. Under et hvilket som helst gitt tidspunkt kan klienter som kontakter systemet under oppgradering bli betjent av enten den gamle eller den nye versjon, men \underline{aldri} begge to. Dessuten må oppgraderingen skje på atomisk vis
  \item \textbf{Naturtro testmiljø}. Testmiljø må rekonstruere realistiske omgivelser som produksjonsmiljø trolig kan utsettes for
\end{itemize}

\cite{dumitracs2009upgrades} sitt bidrag til å automatisering av applikasjonsoppgraderinger er et system der oppdateringer kan utføres online og samtidig inkludere komplekse endringer i dataformatet. I Imago betegner begrepet ''univers'' en distinkt logisk mengde av tjenernoder som eksisterer i form av prosesser som kjører i fysiske datamaskiner eller prosesser i virtualiserte datamaskiner. Image definerer to slike univers, ett for det eksisterende systemet, og ett der den nye versjonen av systemet startes opp og kjører parallellt med den gamle versjonen av systemet. Baktanken med denne arkitekturen er å forhindre versjonsløpsbetingelser (''mixed version race''), noe som \cite{dumitras2010upgrade} viser medfører uante konsekvenser i systemets oppførsel i form av nye feil som oppstår på grunn av uvented kommunikasjon mellom en oppgradert versjon av ett lag i systemet (for eksempel frontend-kode) og en gammel versjon av ett annet lag, som for eksempel datalaget/datamodellen, representert ved mengden av databasetjenere.

\subsection{Diskusjon}
% Kilde: Why Do Upgrades Fail? v/ Tudor Dumitras
En vesentlig antakelse som ligger til grunn for Imago er at det distribuerte systemet har ett aksesspunkt, eller API, for innkommende forespørsler fra brukere, og én eneste metode applikasjonstjenere kommuniserer med databasetjenere/datamodellen på.

% Styrker og svakheter : 2-2
Imago kan garantere fraværet av versjonsløpsbetingelser til gjengjeld for en vesentlig, dog midliertidig økning i bruk av lagringsressurser og beregningsressurser. KVolve har ennå til gode å bli implementert og testet ut på et større distribuert system \citep{saur2016}, men følger det samme udelbarhetsprinsippet som Imago.

% Fra Gramoli
En svakhet med udelbar oppgradering er at oppgraderingsprosessen kan aldri termineres hvis systemet samtidig mottar mange datamanipuleringsforespørsler mens oppgraderingen kjører, fordi en oppdatering øyeblikkelig fører til at alle tuplene som utgjør datamodellens nåværende tilstand må leses på nytt. Derfor avbryter og utsetter Imago skriveforespørsler (\texttt{CREATE}, \texttt{UPDATE}, og \texttt{DELETE}) under oppgraderingsprosessen enten ved å blokkere dem direkte eller ved kun å tillate lesespørringer. For å sikre den atomiske oppgraderingen er det nødvendig at utenforstående operasjoner ikke kan forstyrre datamigrasjoner forbundet til endring av applikasjonens oppførsel.

 % Trenger mer stoff, to delkapitler til
\section{KVolve: Evolusjon av datamodeller uten nedetid}

Den første artikkelen som presenteres i dette kapitlet handler om levende oppdatering av datamodellen til høytilgjengelige webapplikasjoner uten tjenesteavbrudd eller ytelsesdemping, der et nøkkel-verdi-datalager (her representert ved Redis) brukes til å persistere applikasjonens data. Selv om nøkkel-verdi-datalagre er skjemaløse, påfører webapplikasjonen datamodellen gjerne en logisk aggregatstruktur realisert med spesielle dataformat, som protokollbufre, Avro - objekt eller JSON - dittoer \citep{saur2016}.

Nye funksjonelle krav som spesifiseres for webapplikasjoner medfører ofte endringer i datamodellen.  Typiske endringer i aggregatet inkluderer fjerning av, gi nytt navn til eller tillegging av ett eller flere attributter. Dataformatet til aggregatet kan også bli endret fra for eksempel JSON til Apache Thrift. Sistnevnte innehar for øvrig støtte for versjonering og sporing av ''dataformatet''.

Den typiske metoden å implementere slike oppdateringer i NoSQL - datamodellen er å migrere dataene fra den gamle datamodellen, slå av samtlige noder i tjenesten, installere oppdateringen, konvertere dataene til å passe den nye modellen, og deretter importere dataene inn i den oppdaterte databasen. Dette medfører selvsagt nedetid i systemet, som tradisjonelt sett kan bortdefineres (jamfør kapittel 2.1), da vedlikeholdsarbeid gjøres på et tidspunkt ingen behøver systemets tjenester.

I en høytilgjengelig webapplikasjon som betjener forespørsler nærmest døgnet rundt er denne stopp-og-restart-strategien uholdbar. At konvertering av data til å konformere til et nytt aggregatformat tar tid er også problematisk. Ved å opprettholde bakoverkompabilitet på dataformatet kan systemets utviklere trygt oppgradere datamodellen med en manuell, rullerende teknikk, men da vil man samtidig legge kraftige begrenser på hvordan applikasjonen videreutvikles.

\subsection{Løsning}
KVolve, hvis navn kommer fra frasen \emph{Key-value store evolution}, er en utvidelse av det populære databasesystemet Redis. Det er implementert i C i form av et modulært bibliotek kompilert sammen med Redis. Kildekoden til dette selvstendige databasesystemet ble publisert på GitHub den 18. september 2016 \footnote{Tilgjengelig på Github, \url{https://github.com/plum-umd/kvolve}}, og benytter kildekodeversjon redis-2.8.17 av Redis i sin implementasjon. KVolve fungerer som et eget databasesystem som tilbyr støtte for levende skjemaoppgradering gjennom lat datamigrering (eng. ''lazy migration'') \citep{saur2016}.

Systemet eksponerer aggregatformatets versjon for applikasjonen. Data konverteres ikke på ivrig vis ved for eksempel å iterere over nøklene i databasen og endre både dem (hvis nødvendig) og deres verdier. Formatkonvertering av nøkler og verdier utføres når applikasjonen gjør oppslag på dem. KVolve sporer versjonene til de forskjellige dataobjektene ved hjelp av prefikser i nøklene. For hvert dataobjekt lagres en ''version tag'' som i tilfeller der dataobjekt ikke har blitt migrert kan være eldre enn dets logiske versjon \citep{saur2016}.

Idet applikasjonen kopler til KVolve må den indikere hvilken logisk versjon av datamodellen den forventer å få. Tilkoplingen tillates hvis den oppgitte versjonen tilsvarer den nyeste KVolve har opprettet. Ved oppgradering av datamodellen må logisk nok også selve applikasjonen oppgraderes for å kunne behandle den nye versjonen. Dette kan for eksempel gjøres på manuelt, rullerende vis. For å kunne ta i bruke dette systemet kreves det ingen manuell versjonshandtering i applikasjonen da datamodellen som applikasjonen KVolve lagrer data til til enhver kan se en logisk konsistent versjon av datamodellen \citep{saur2016}.

% TODO: Beskriv state transformation function i dsu-kapitlet
Med dette systemet kan utviklere definere en oppgraderingsspesifikasjon som består av en mengde transformasjonsfunksjoner, som likner på det \cite{hicks2005dsu} kaller for ''state transformation functions'', i forbindelse med dynamisk oppdatering. Michael Hicks er for øvrig en av medforfatterne av denne artikkelen. Oppgraderingsspesifikasjonen definerer hvordan eksisterende data må transformeres for å passe den nye datamodellen. Disse transformasjonsfunksjonene kompileres til en delt objekt - fil som KVolve kan lese inn ved hjelp av I/O - grensesnittet til Redis. Når datamodellen er oppdatert, blir dette delte objektet persistert til disk i Redis.

Selve datatransformasjonen følger en lat strategi, altså blir data transformert i tråd med spesifikasjonen først når den spørres etter i applikasjonen. Lat datamigrering er også en kjent strategi som kan benyttes ved rullerende oppgraderinger av datamodeller, der dataobjekter som aksesseres konverteres til det nye aggregatformatet når spørringer inntreffer. Denne konverteringen gjøres da for hver enkelt spørring, og legger en vesentlig demper på spørreytelsen tilø databasen. Late beregningsstrategier står sentralt i det funksjonelle programmeringsparadigmet, der de utgjør en vesentlig faktor for ytelsen til språk som Haskell, Scala og Erlang.

KVolve definerer to typer oppdateringer av formater i datamodellen: 1: Oppdatering kun av \\ nøkkelformatet og 2: oppdatering av nøkkelformatet og verdiformatet. Ved sistnevnte påkalles transformasjonsfunksjonen med referanse til en gammel nøkkel i form av en streng og en referanse til en mengde binærdata, hvilket representerer verdien strengnøkkelen peker på i databasen, som argumenter. Transformasjonsfunksjonen bruker disse referansene til å endre nøkkelen og dets assosierte verdi til å samsvare med den nye versjonen av datamodellen. Figur 1 til og med 5 illustrerer et eksempel på en datamodelloppgradering \citep{saur2016}.

\subsection{Evaluering av løsning}
Artikkelens eksperimentelle resultater kan kort oppsummeres som følger: Ved hjelp av ytelsesmålingsverktøy inebygd i Redis ble det avdekket lite ekstra kostnad i ytelse ved normale databaseoperasjoner, og versjonslagring og dataoppdatering medfører et tillegg i lagringsforbruk på inntil 15~\% (128.6 MB kontra 112.1 MB). I tillegg ble en test utført der filsystemet brukt til å lagre data med, RedisFS, ble oppdatert. I denne oppdateringen utføres også navneendring på enkelte nøkler og komprimering av dataverdier. Den late datatransformasjonsstrategien utklasser stopp-og-restart-strategien i tidsbruk. Offline datamigrasjon brukte 12 sekunder på nevnte oppdatering. Versjonskontrolleringen i KVolve medfører naturlig nok også litt ekstra minnebruk. Evalueringene ble utført på én enkel datamaskin med 24 prosessorkjerner og 32 GB RAM, med operativsystem Red Hat Enterprise Linux, versjon 6.5.

\emph{Redis-bench} er et innebygd benchmarking - verktøy i Redis. Redis-bench emulerer en klientprosess som sender forespørsler til Redis-databasetjeneren. For denne ytelsestestingen av KVolve er antallet nøkler i databasen konfigurert til én million, og antallet operasjoner til fem millioner, slik at testen strekker over lang tid og mange forespørsler. Dermed får man et klarere bilde av det langsiktige ressursoverheadet det dynamiske skjemaoppgraderingsrammeverket påfører applikasjonen, og i tillegg blir testscenariet så realistisk som mulig.

For å få stabile testresultater som eliminerer eksepsjonelle enkeltresultater har benchmarkingen blitt kjørt flere ganger og tallene \cite{saur2016} lister opp i tabell 1 av artikkelen er mediantider målt med en nøyaktighet på hundredels sekundet, beregnet utifra i alt 11 forsøk.

Ytelsespåvirkningen (målt i antall spørringer per sekund) til to forskjellige applikasjonsoppgraderinger har blitt testet. I det ene scenariet har RedisFS, et filsystem hvis metadata som inoder og kataloger og filsystemdata er lagret i Redis, blitt oppgradert fra versjon \texttt{redisfs.5} til \texttt{redisfs.7}. Tre forskjellige metoder har blitt brukt i oppgraderingene. I den ivrige metoden blir forespørselstrafikken til Redis avbrutt helt, navngivning av nøkler utføres ved behov og data migrereres på manuelt vis. KVolve sin oppgraderingsmetode, der tjenesteforespørsler blir behandlet mens oppgraderingen utføres, er blitt realisert med kodeendringsverktøyet Kitsune, som også er blitt utviklet ved universitet ved Maryland av blant annet artikkelens hovedforfatter. Den tredje metoden er den manuelle, late migrasjonsmetoden beskrevet i kapittel 12 av \cite{sadalage2013}.

I det andre oppdateringseksperimentet ble et sosialt medium implementert ved hjelp grafbiblioteket Amico (det italienske ordet for ''venn''). I dette systemet ble Amico - komponenten oppgradert fra versjon \texttt{1.2.0} til \texttt{2.0.0}, både ved manuelle datamigrering og ved hjelp av KVolve. Med datamigrasjon på vanlig, ivrig vis var applikasjonen utilgjengelig i totalt ett minutt og 27 sekunder, mot ett sekund for KVolve \citep{saur2016}. Figur 7 illustrerer disse ytelsesforskjellene tydelig og kvantitativt.

\subsection{Diskusjon}
Til tross for artikkelens praktiske tilnærming er den publiserte kildekoden kun ment for å være en proof-of-concept - implementasjon av levende skjemaoppgradering. Ved evaluering av systemet har det blitt foretatt både mikro-benchmarking og marko-benchmarking, altså har man testet oppgraderingsmekanikkens ytelsespåvirking både på databasens indre ytelse og dets innvirkning på en kjørende webapplikasjon i sin helhet.

Evalueringsseksjonen fokuserer ikke på å sammenlikne KVolve med andre systemer som organiserer skjemaendringer på dynamisk vis. Dette kan ha noe med at samhandling mellom datammodellendringer og applikasjonslogikken er en oppgave som tradisjonelt overlates til den enkelte applikasjonsvedlikeholder. Derfor er det svært få systemer KVolve kan sammenliknes med. Ett sammenlignbart system \cite{saur2016} nevner er Googles F1, som har en asynkron protokoll som kan legge til å fjerne tabeller kolnner og indekser på dynamisk vis slik at F1 har tilgang til datamodellen under oppgradering. Her blir linearisering av oppdateringer handtert ved hjelp av fysiske klokker levert av TrueTime API-et til Google. F1 støtter imidlertid ikke endring av ProtoBuf - formatet som lagres i dets kolonner.

De fem første figurene i artikkelen gir et godt bilde av en eksempelapplikajson som henter inspirajson fra \cite{sadalage2013} og hvordan KVolve oppfører seg med de gitte objektformater. Figur 6 og 7 illustrerer tydelig ytelsesfordelen KVolve gir sammenliknet med konvensjonell lat datamigrasjon og ivrig, rullerende oppdatering.

Et annet poeng til denne artikkelen er hvilken type problem med levende oppgradering av databaser den prøver å løse. KVolve behandler oppgradering av datamodeller, ikke hele databaseapplikasjoner, slik som Imago kan få til. Det påpekes i sistnevntes disfavør at mens den realiserer atomisk oppgradering av applikasjoner så er det på bekostning av en vesentlig økning, om enn midlertidig, ressursbruk i form av dataduplisering og dobbelt opp av noder.

Testene referert til i forrige delkapittel er ikke gjort i et reelt distribuert system med forskjellige maskiner på forskjellige datasentre rundt om i verden, men på én enkelt datamaskin som bruker loopback-grensesnittet (localhost). Testresultatene gir altså ikke et bilde av hvordan KVolve kan yte i et moderne produksjonsmiljø, et aspekt artikkelens konklusjon gjerne har lyst til å finne ut av ved å få KVolve implementert på Redis Cluster.



\cleardoublepage